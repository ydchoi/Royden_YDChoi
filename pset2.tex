\documentclass{article} % For LaTeX2e
\usepackage{nips14submit_e,times}
\usepackage{amsmath}
\usepackage{amsthm}
\usepackage{amssymb}
\usepackage{mathtools}
\usepackage{hyperref}
\usepackage{url}
\usepackage{algorithm}
\usepackage[noend]{algpseudocode}
%\documentstyle[nips14submit_09,times,art10]{article} % For LaTeX 2.09

\usepackage{mathrsfs}
\usepackage{graphicx}
\usepackage{caption}
\usepackage{subcaption}

\def\eQb#1\eQe{\begin{eqnarray*}#1\end{eqnarray*}}
\def\eQnb#1\eQne{\begin{eqnarray}#1\end{eqnarray}}
\providecommand{\e}[1]{\ensuremath{\times 10^{#1}}}
\providecommand{\pb}[0]{\pagebreak}


\def\Qb#1\Qe{\begin{question}#1\end{question}}
\def\Sb#1\Se{\begin{solution}#1\end{solution}}

\newenvironment{claim}[1]{\par\noindent\underline{Claim:}\space#1}{}
\newtheoremstyle{quest}{\topsep}{\topsep}{}{}{\bfseries}{}{ }{\thmname{#1}\thmnote{ #3}.}
\theoremstyle{quest}
\newtheorem*{definition}{Definition}
\newtheorem*{theorem}{Theorem}
\newtheorem*{lemma}{Lemma}
\newtheorem*{question}{Question}
\newtheorem*{preposition}{Preposition}
\newtheorem*{exercise}{Exercise}
\newtheorem*{challengeproblem}{Challenge Problem}
\newtheorem*{solution}{Solution}
\newtheorem*{remark}{Remark}
\usepackage{verbatimbox}
\usepackage{listings}

\title{Real Variables: \\
Problem Set II}


\author{
Youngduck Choi \\
Courant Institute of Mathematical Sciences \\
New York University \\
\texttt{yc1104@nyu.edu} \\
}


% The \author macro works with any number of authors. There are two commands
% used to separate the names and addresses of multiple authors: \And and \AND.
%
% Using \And between authors leaves it to \LaTeX{} to determine where to break
% the lines. Using \AND forces a linebreak at that point. So, if \LaTeX{}
% puts 3 of 4 authors names on the first line, and the last on the second
% line, try using \AND instead of \And before the third author name.

\newcommand{\fix}{\marginpar{FIX}}
\newcommand{\new}{\marginpar{NEW}}

\nipsfinalcopy % Uncomment for camera-ready version

\begin{document}


\maketitle

\begin{abstract}
This work contains solutions to the problem set II of Real Variables 2015 at NYU.
\end{abstract}

\section{Solutions}

\begin{question}[3. Royden 2.29]
\end{question}
\begin{solution}
\textbf{(i)} 
Let $X$ be any set of real numbers, and $R$ be the relation defined by the rational
equivalence. For $x \in X$, we have $x - x = 0$. Hence, the rational equivalence is reflexive.
Let $(x,y) \in R$, then we have $x - y \in \mathbb{Q}$. As a negative of a rational number
is rational, we have $y-x \in \mathbb{Q}$ and $(y,x) \in R$. Hence, the rational equivalence is
symmetric. Let $(x,y), (y,z) \in R$. As a sum of two rationals is rational, we have
$x - y + y - z$, which is $x-z$, is rational, and $(x,z) \in R$. Hence, the rational equivalence
is transitive. Therefore, the rational equivalence is an equivalence relation.

\smallskip

\textbf{(ii)} The partition of $\mathbb{Q}$, induced by the rational equivalence, is simply
$\{ \mathbb{Q} \}$. Hence, $\{ 0 \}$ is a set that consists of exactly one member of each
equivalence class. Therefore, $\{ 0 \}$ is an explicit choice set of the rational equivalence.

\smallskip

\textbf{(iii)} We define two numbers to be irrationally equivalent provided their difference
is irrational. Let $x \in \mathbb{R}$. As $x - x = 0$ and $0$ is a rational number, the relation
defined fails to be reflexive. Hence, The relation is not an equivalence relation on 
$\mathbb{R}$. The same 
argument holds with $x \in \mathbb{Q}$, and the relation is not an equivalence relation on 
$\mathbb{Q}$ as well. $\qed$
\end{solution}

\bigskip

\begin{question}[3. Royden 2.38]
\end{question}
\begin{solution}
Let $f:[a,b] \to \mathbb{R}$ be Lipschitz with the associated Lipschitz constant $c$,
and let $E_0 \in [a,b]$ such that $\mathrm{m}(E_0) = 0$.
Fix $\epsilon > 0$. As $\mathrm{m}(E_0) = 0$, we have a countable collection of disjoint 
open intervals $\{ I_k \}_{k=1}^{\infty}$ such that $E \subseteq 
\cup_{k=1}^{\infty} I_k$ and $\sum_{k=1}^{\infty}
\mathrm{m}(I_k) < \dfrac{\epsilon}{c}$. Since $E \subseteq \cup_{k=1}^{\infty} 
I_k$, we have $f(E_0) \subseteq \cup_{k=1}^{\infty} f(I_k)$. By the monotonicity
of measure, and Lipshitz property of $f$, we obtain
\eQb
\mathrm{m}(f(E_0)) \leq \sum_{k=1}^{\infty} \mathrm{m}(f(I_k)) 
\leq c\sum_{k=1}^{\infty} \mathrm{m}(I_k) = \epsilon.
\eQe
Since $\epsilon$ is arbitrary, we have $\mathrm{m}(f(E_0)) = 0$. Therefore, we have shown that
a Lipschitz function maps a set of zero measure on to a set of measure zero. 
\end{solution}

\bigskip

\begin{question}[3. Royden 3.1]
\end{question}
\begin{solution}
Let $f$ and $g$ are continuous functions on $[a,b]$. Assume that $f=g$ a.e. In other words,
$f=g$ on $[a,b] \setminus E_0$, where $\mathrm{m}(E_0) = 0$. Let $x \in E_0$, and
fix $\epsilon > 0$. By the continuity of $f$ and $g$, we have
$\delta_f$ and $\delta_g$ such that
\eQnb
|x - x^{\prime}| < \delta_f &\implies& |f(x) - f(x^{\prime})| < \dfrac{\epsilon}{2} \nonumber \\
|x - x^{\prime}| < \delta_g &\implies& |g(x) - g(x^{\prime})| < \dfrac{\epsilon}{2}
\eQne
Now, consider the set $B(x,\min(\delta_f,\delta_g)) \cap [a,b]$, where $B$ denotes a ball with a center
and radius. As $E_0$ is a zero measure set, there exists $x^{*}$ in $B(x, \min(\delta_f, \delta_g)) \cap
[a,b]$ such that $f(x^{*}) = g(x^{*})$. Furthermore, by $(1)$, we have that 
$|f(x) - f(x^{*})| < \dfrac{\epsilon}{2}$ and $|g(x) - g(x^{*})| < \dfrac{\epsilon}{2}$.
Consequently, by the trinagle inequality, we have 
\eQb
|f(x) - g(x)| &\leq& |f(x) - f(x^{*})| + |g(x) - g(x^{*})| + |f(x^{*}) - g(x^{*})| = \epsilon.
\eQe
Since $\epsilon$ was arbitrary, we have shown that for $x \in E_0$, we have $f(x) = g(x)$. Therefore,
$f=g$ on $[a,b]$ holds. $\qed$
\end{solution}

\bigskip


\begin{question}[4. Royden 3.5]
\end{question}
\begin{solution}
Assume that the function $f$ is defined on a measurable domain $E$ and has a property that
$\{ x \in E \> | \> f(x) > c \}$ is measurable for each rational number $c$. Let $r \in \mathbb{R} 
\setminus \mathbb{Q}$. Consider the set $\{ x \in E \> | \> f(x) > r \}$. Notice that
\eQb
\{ x \in E \> | \> f(x) > c\} &=& \bigcup_{k=1}^{\infty}
\{ x \in E \> | \> f(x) \geq c + \dfrac{1}{k} \}.
\eQe
By the density of the rationals, we can choose a sequence of rationals,
$\{ c_k \}$ such that for each $k$, we have $c_k \in \mathbb{Q}$ and 
$c_k \in (c, c + \dfrac{1}{k})$. In particular, we have that 
\eQb
\{ x \in E \> | \> f(x) > c \}
&=& \bigcup_{k=1}^{\infty} \{ x \in E \> | \> f(x) \geq c_k \}.
\eQe
As $\{ c_k \}$ is a rational sequence, 
$\{ x \in E \> | \> f(x) \geq c_k \}$ is measurable for all $k$, and 
$\{ x \in E \> | \> f(x) > c \}$ is measurable, as a countable union of measurable sets is measurable.
Since $r$ is an arbitrary irrational, we have shown that 
$\{ x \in E \> | \> f(x) > a \}$ is measurable for any $a \in \mathbb{R}$. Therefore, $f$ is 
measurable. $\qed$

\end{solution}

\bigskip

\begin{question}[5. Royden 3.7]
\end{question}
\begin{solution}
Let $f$ be a function defined on a measurable set $E$. We wish to show that $f$
is measurable if and only if an inverse image of any Borel set is measurable. We
denote the Borel $\sigma$-algebra as $\mathscr{B}$.\\

Assume that an inverse image of any borel set is measurable. Then, as 
the $(c,\infty)$ is a borel set for any $c$, we have that $f^{-1}((c,\infty))$,
which can be re-written as $\{ x \in E \> | \> f(x) > c \}$, is measurable
for any $c$. This is precisely the definition of a measurable function. Hence, $f$
is measurable. 

\smallskip

Assume that $f$ is measurable. Note that
\eQb
\bigcup_{k=1}^{\infty} f^{-1}(E_k) &=& f^{-1}(\bigcup_{k=1}^{\infty} E_k ) \\
\bigcap_{k=1}^{\infty} f^{-1}(E_k) &=& f^{-1}(\bigcap_{k=1}^{\infty} E_k ) 
\eQe
Hence, $f^{-1}(B)$ is measurable for $B \in \mathscr{B}$. 


\end{solution}

\bigskip

\begin{question}[6. Royden 3.9]
\end{question}
\begin{solution}
Let $\{ f_n \}$ be a sequence of measurable functions defined on a measurable set $E$.
Let $E_0 = \{ x \in E \> | \> \{ f_{n}(x) \} \text{ converges}\}$. 
By the Cauchy Criterion of real sequences, we can re-characterize $E_0$ as follows:  
\eQb
E_0 &=& \{ x \in E \> | \> \forall K \in \mathbb{N}, \exists N \in \mathbb{N}
\text{ such that } |f_{n}(x) - f_{m}(x)| < \dfrac{1}{K} \text{ for } 
n,m \geq N \} \\
&=& \bigcap_{K = 1}^{\infty} \bigcup_{N=1}^{\infty} \{ x \in E \> | \>
|f_{n}(x) - f_m(x)| < \dfrac{1}{K}
\text{ for } n,m \geq N \}.
\eQe

We have that for a measurable function $f$ and $g$,  
$|f-g|$ is measurable. Hence,  $|f_n - f_m |$ is measurable. Consequently,
$\{ x \in E \> | \> | f_{n}(x) - f_{m}(x) | < \dfrac{1}{K} \text{ for }
n,m \geq N \}$ is a measurable set for all $K$ and $N$. Then, $E_0$ is 
a countable intersection of countable union of measurable sets, and thus
is measurable. We have shown that $E_0$ is measurable. $\qed$


\end{solution}


\bigskip
\end{document}
