\documentclass{article} % For LaTeX2e
\usepackage{nips14submit_e,times}
\usepackage{amsmath}
\usepackage{amsthm}
\usepackage{amssymb}
\usepackage{mathtools}
\usepackage{hyperref}
\usepackage{url}
\usepackage{algorithm}
\usepackage[noend]{algpseudocode}
%\documentstyle[nips14submit_09,times,art10]{article} % For LaTeX 2.09

\usepackage{mathrsfs}
\usepackage{graphicx}
\usepackage{caption}
\usepackage{subcaption}

\def\eQb#1\eQe{\begin{eqnarray*}#1\end{eqnarray*}}
\def\eQnb#1\eQne{\begin{eqnarray}#1\end{eqnarray}}
\providecommand{\e}[1]{\ensuremath{\times 10^{#1}}}
\providecommand{\pb}[0]{\pagebreak}


\def\Qb#1\Qe{\begin{question}#1\end{question}}
\def\Sb#1\Se{\begin{solution}#1\end{solution}}

\newenvironment{claim}[1]{\par\noindent\underline{Claim:}\space#1}{}
\newtheoremstyle{quest}{\topsep}{\topsep}{}{}{\bfseries}{}{ }{\thmname{#1}\thmnote{ #3}.}
\theoremstyle{quest}
\newtheorem*{definition}{Definition}
\newtheorem*{theorem}{Theorem}
\newtheorem*{lemma}{Lemma}
\newtheorem*{question}{Question}
\newtheorem*{preposition}{Preposition}
\newtheorem*{exercise}{Exercise}
\newtheorem*{challengeproblem}{Challenge Problem}
\newtheorem*{solution}{Solution}
\newtheorem*{remark}{Remark}
\usepackage{verbatimbox}
\usepackage{listings}

\title{Real Variables: \\
Problem Set VI}


\author{
Youngduck Choi \\
Courant Institute of Mathematical Sciences \\
New York University \\
\texttt{yc1104@nyu.edu} \\
}


% The \author macro works with any number of authors. There are two commands
% used to separate the names and addresses of multiple authors: \And and \AND.
%
% Using \And between authors leaves it to \LaTeX{} to determine where to break
% the lines. Using \AND forces a linebreak at that point. So, if \LaTeX{}
% puts 3 of 4 authors names on the first line, and the last on the second
% line, try using \AND instead of \And before the third author name.

\newcommand{\fix}{\marginpar{FIX}}
\newcommand{\new}{\marginpar{NEW}}

\nipsfinalcopy % Uncomment for camera-ready version

\begin{document}


\maketitle

\begin{abstract}
This work contains solutions to the problem set 
VI of Real Variables 2015 at NYU.
\end{abstract}

\section{Solutions}

\begin{question}[9.10]
\end{question}
\begin{solution}
Let $\{X_n, \rho_n \}_{n=1}^{\infty}$ be a countable collection
of metric spaces. We now define 
$( \prod_{n=1}^{\infty} X_n , p_*) = (X,p_*)$ such that for $x,y \in 
X$,
\eQb
p_*(x,y) &=& \sum_{n=1}^{\infty} \dfrac{1}{2^n} \cdot
\dfrac{p_n(x_n,y_n)}{1+p_n(x_n,y_n)}. 
\eQe
First, we can show that $p_*$ is well-defined via
comparison test with the series $\sum_{n=1}^{\infty} \dfrac{1}{2^n}$,
as $0 \leq \dfrac{p_n(x_n,y_n)}{1+p_n(x_n,y_n)} \leq 1$ for all $n$. \\

\smallskip

As $p_n(x_n,y_n) \geq 0$ for all $n$, we have $p_*(x,y) \geq 0$ for all 
$x,y \in X$. If $p_*(x,y) = 0$, then $p_n(x_n,y_n) = 0$ for all $n$. 
As each $p_n$ is a metric space $x_n = y_n$ for all $n$. Therefore, 
$x = y$. If $x = y$, then $x_n = y_n$ for all $n$. As each $p_n$
is a metirc space, $p_n(x_n,
y_n) = 0$ for all $n$. Therefore, $p_*(x,y) = 0$. \\

\smallskip
Since $p_n(x_n,y_n) = p_n(y_n,x_n)$ for all $n$, for $x,y \in X$, we
\eQb
p_*(x,y) &=& \sum_{n=1}^{\infty} \dfrac{1}{2^n} \cdot
\dfrac{p_n(x_n,y_n)}{1+p_n(x_n,y_n)} \\
&=& \sum_{n=1}^{\infty} \dfrac{1}{2^n} \cdot
\dfrac{p_n(y_n,x_n)}{1+p_n(y_n,x_n)} \\
&=& p_*(y,x).
\eQe

Let $x,y,z \in X$. By the problem $6$ and the triangle inequality
of each metric space $X_n$, which gives $p_n(x_n,z_n) \leq
p_n(x_n,y_n) + p_n(y_n,z_n)$ for each $n$,  we have
\eQb
\dfrac{p_n(x_n,z_n)}{1 + p_n(x_n,z_n)} &\leq& 
\dfrac{p_n(x_n,y_n)}{1 + p_n(x_n,y_n)} + 
\dfrac{p_n(y_n,z_n)}{1 + p_n(y_n,z_n)}, \\ 
\eQe 
for all $n$. Hence, we have
\eQb
\sum_{n=1}^{\infty} \dfrac{p_n(x_n,z_n)}{1 + p_n(x_n,z_n)} &\leq& 
\sum_{n=1}^{\infty} \dfrac{p_n(x_n,y_n)}{1 + p_n(x_n,y_n)} + 
\dfrac{p_n(y_n,z_n)}{1 + p_n(y_n,z_n)}, \\ 
\eQe
which can be written as
\eQb
p_*(x,z) &\leq& p_*(x,y) + p_*(y,z).
\eQe
Therefore, we have shown that all required properties of a metric space
hold for $(X,p_*)$. $(X,p_*)$ is a metric space. $\qed$

\end{solution}

\bigskip

\begin{question}[9.20]
\end{question}
\begin{solution}
Let $E$ be a subset of a metric space $X$, and let $\text{int}E$ 
be the interior
of $E$. We first show that $\text{int}E \subseteq E$.
If $x \in X \setminus E$, then every ball of $x$ contains
a point in $X\setminus E$. Hence, $x \in E$. Therefore, $\text{int}E 
\subseteq E$. 

Now, we wish to show that $\text{int}E$ is open. For the first case,
assume that $E = \text{int}E$. 
Let $ x \in \text{int} E$. Since $x$ is an interior point of 
$E$, there exists an open ball $B(x,r)$ contained in $E$. 
Since $E = \text{int}E$, the open ball $B(x,r)$ is contained in 
$\text{int}E$ as well. Hence, $\text{int}E$ is open in this case.
For the remaining case, assume that $E \setminus \text{int}E 
\neq \emptyset$. Let $x \in \text{int}E$. Since $x$ is an interior
point of $E$, there exists an open ball $B(x,r)$ contained in $E$.
Suppose that there exists $y \in B(x,r) \cap E \setminus \text{int}E$.
Then, we have $d(x,y) < r$. Consider $B(y,r - d(x,y))$, which is valid
since $r - d(x,y) > 0$. 
By the triangle inequality, for any point 
$z \in B(y,r-d(x,y))$,
\eQb
d(x,z) &\leq& d(x,y) + d(y,z) \\
&<& r. 
\eQe
Hence, $B(y,r-d(x,y))$ is an open ball contained in $B(x,r)$, which is
again contained $E$, which contradicts
the fact that $y \in E \setminus \text{int}E$. 
Hence, $B(x,r)$ is contained in $\text{int}E$. Therefore, $\text{int}E$
is open. As we covered all cases, $\text{int}E$ for any 
subset $E$ of a metric space $X$ is open. \\

\smallskip 

Assume $E$ is open. Let $ x \in E$. As $E$ is open,
there exists an open ball around $x$ contained in $E$. Therefore,
$x \in \text{int}E$. Hence, $E \subseteq \text{int}E$. As we have
$\text{int}E \subseteq E$ from above, we have shown that 
$E = \text{int}E$. 

Assume $E = \text{int}E$. Let $x \in E$. Then, as $E = \text{int}E$,
$x \in \text{int}E$. By the definition of interior point, there exists
an open ball around $x$ contained in $E$. Hence, $E$ is open.
$\qed$
\end{solution}

\bigskip

\begin{question}[9.32]
\end{question}
\begin{solution}
\textbf{(a)}
Let $\{x_n\}$ be a sequence from $X$, such that $x_n \to x$.
Fix $\epsilon > 0$. As $x_n \to x$, there exists an index $N$, such that
$\rho(x_n,x) < \epsilon$ for $n \geq N$. From the triangle inequality,
it follows that
\eQb
|f(x_n) - f(x)| &=& |\text{dist}(x_n,E) - \text{dist}(x,E)| \\
&=& |\inf\{\rho(x_n,y) \> | \> y \in E\} - 
\inf\{\rho(x,y) \> | \> y \in E\}|  \\
&\leq& \rho(x_n,x) \\
&<& \epsilon,
\eQe  
for $n \geq N$.
Since $\epsilon$ was arbitrary, $f(x_n) \to f(x)$.
Therefore, as $f$ satisfies the sequential characterization of continuity,
$f$ is continuous. 


\smallskip

\textbf{(b)} 
Let $x \in \overline{E}$. By the definition of closure, there exists
a sequence $\{ x_n \}$ from $E$ such that $x_n \to x$. 
Since $\text{dist}$ is continuous, we have
$ \text{dist}(x_n,E) \to \text{dist}(x,E)$. As 
$x_n \in E$, it follows that $\text{dist}(x_n,E) = 0$ 
for all $n$. Therefore, we have $\text{dist}(x,E) = 0$. Hence, we obtain
\eQb 
\overline{E} &\subseteq&  \{x \in X \>\> | \>\>  \text{dist}(x,E) = 0\}. 
\eQe
Let $x \in \{ x \in X \>\> | \>\> \text{dist}(x,E) = 0\}$. Then, for 
all $n$, there exists $y \in E$ such that $\rho(x,y) > \dfrac{1}{n}$,
which we label as $x_n$. Then, $\{x_n \}$ is a sequence from $E$
such that $x_n \to x$. Hence, $x \in \bar{E}$. Consequently,
we obtain 
\eQb 
 \{x \in X \>\> | \>\>  \text{dist}(x,E) = 0\} &=& \overline{E}, 
\eQe
as desired. $\qed$

\end{solution}

\bigskip

\begin{question}[9.43]
\end{question}
\begin{solution}

\end{solution}

\bigskip

\begin{question}[9.72]
\end{question}
\begin{solution}
Assume $A \cap B \neq \emptyset$. Then, there exists $x \in A \cap B$.
Since $\rho(x,x) = 0$, we have $\text{dist}(A,B) = 0$. By contrapositive,
we have shown that if $\text{dist}(A,B) > 0$, then $A \cap B = \emptyset$.
\end{solution}

\bigskip

\begin{question}[9.77]
\end{question}
\begin{solution}
Let $X$ and $Y$ be separable metric spaces. Consider the standard
product metric on $X \times Y$. Then, there exist 
a countable dense subset $D_X$ in $X$ and countable dense 
subset $D_Y$ in $Y$. Observe that $D_X \times D_Y$ is countable. 
We claim that $D_X \times D_Y$ is a dense
subset in $X \times Y$. Therefore, $X \times Y$ is separable. $\qed$
\end{solution}


\end{document} 
