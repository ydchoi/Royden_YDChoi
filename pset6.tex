\documentclass{article} % For LaTeX2e
\usepackage{nips14submit_e,times}
\usepackage{amsmath}
\usepackage{amsthm}
\usepackage{amssymb}
\usepackage{mathtools}
\usepackage{hyperref}
\usepackage{url}
\usepackage{algorithm}
\usepackage[noend]{algpseudocode}
%\documentstyle[nips14submit_09,times,art10]{article} % For LaTeX 2.09

\usepackage{mathrsfs}
\usepackage{graphicx}
\usepackage{caption}
\usepackage{subcaption}

\def\eQb#1\eQe{\begin{eqnarray*}#1\end{eqnarray*}}
\def\eQnb#1\eQne{\begin{eqnarray}#1\end{eqnarray}}
\providecommand{\e}[1]{\ensuremath{\times 10^{#1}}}
\providecommand{\pb}[0]{\pagebreak}


\def\Qb#1\Qe{\begin{question}#1\end{question}}
\def\Sb#1\Se{\begin{solution}#1\end{solution}}

\newenvironment{claim}[1]{\par\noindent\underline{Claim:}\space#1}{}
\newtheoremstyle{quest}{\topsep}{\topsep}{}{}{\bfseries}{}{ }{\thmname{#1}\thmnote{ #3}.}
\theoremstyle{quest}
\newtheorem*{definition}{Definition}
\newtheorem*{theorem}{Theorem}
\newtheorem*{lemma}{Lemma}
\newtheorem*{question}{Question}
\newtheorem*{preposition}{Preposition}
\newtheorem*{exercise}{Exercise}
\newtheorem*{challengeproblem}{Challenge Problem}
\newtheorem*{solution}{Solution}
\newtheorem*{remark}{Remark}
\usepackage{verbatimbox}
\usepackage{listings}

\title{Real Variables: \\
Problem Set VI}


\author{
Youngduck Choi \\
Courant Institute of Mathematical Sciences \\
New York University \\
\texttt{yc1104@nyu.edu} \\
}


% The \author macro works with any number of authors. There are two commands
% used to separate the names and addresses of multiple authors: \And and \AND.
%
% Using \And between authors leaves it to \LaTeX{} to determine where to break
% the lines. Using \AND forces a linebreak at that point. So, if \LaTeX{}
% puts 3 of 4 authors names on the first line, and the last on the second
% line, try using \AND instead of \And before the third author name.

\newcommand{\fix}{\marginpar{FIX}}
\newcommand{\new}{\marginpar{NEW}}

\nipsfinalcopy % Uncomment for camera-ready version

\begin{document}


\maketitle

\begin{abstract}
This work contains solutions to the problem set 
VI of Real Variables 2015 at NYU.
\end{abstract}

\section{Solutions}

\begin{question}[9.10]
\end{question}
\begin{solution}
Let $\{X_n, \rho_n \}_{n=1}^{\infty}$ be a countable collection
of metric spaces. We now define 
$( \prod_{n=1}^{\infty} X_n , p_*) = (X,p_*)$ such that for $x,y \in 
X$,
\eQb
p_*(x,y) &=& \sum_{n=1}^{\infty} \dfrac{1}{2^n} \cdot
\dfrac{p_n(x_n,y_n)}{1+p_n(x_n,y_n)}. 
\eQe
First, we can show that $p_*$ is well-defined via
comparison test with the series $\sum_{n=1}^{\infty} \dfrac{1}{2^n}$,
as $0 \leq \dfrac{p_n(x_n,y_n)}{1+p_n(x_n,y_n)} \leq 1$ for all $n$. \\

\smallskip

As $p_n(x_n,y_n) \geq 0$ for all $n$, we have $p_*(x,y) \geq 0$ for all 
$x,y \in X$. If $p_*(x,y) = 0$, then $p_n(x_n,y_n) = 0$ for all $n$. 
As each $p_n$ is a metric space $x_n = y_n$ for all $n$. Therefore, 
$x = y$. If $x = y$, then $x_n = y_n$ for all $n$. As each $p_n$
is a metirc space, $p_n(x_n,
y_n) = 0$ for all $n$. Therefore, $p_*(x,y) = 0$. \\

\smallskip
Since $p_n(x_n,y_n) = p_n(y_n,x_n)$ for all $n$, for $x,y \in X$, we
\eQb
p_*(x,y) &=& \sum_{n=1}^{\infty} \dfrac{1}{2^n} \cdot
\dfrac{p_n(x_n,y_n)}{1+p_n(x_n,y_n)} \\
&=& \sum_{n=1}^{\infty} \dfrac{1}{2^n} \cdot
\dfrac{p_n(y_n,x_n)}{1+p_n(y_n,x_n)} \\
&=& p_*(y,x).
\eQe

Let $x,y,z \in X$. By the problem $6$ and the triangle inequality
of each metric space $X_n$,  we have
\eQb
\dfrac{p_n(x_n,z_n)}{1 + p_n(x_n,z_n)} &\leq& 
\dfrac{p_n(x_n,y_n)}{1 + p_n(x_n,y_n)} + 
\dfrac{p_n(y_n,z_n)}{1 + p_n(y_n,z_n)} \\ 
\eQe 

\end{solution}

\bigskip

\begin{question}[9.20]
\end{question}
\begin{solution}
\end{solution}

\bigskip

\begin{question}[9.32]
\end{question}
\begin{solution}
\end{solution}

\bigskip


\end{document} 
