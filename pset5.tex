\documentclass{article} % For LaTeX2e
\usepackage{nips14submit_e,times}
\usepackage{amsmath}
\usepackage{amsthm}
\usepackage{amssymb}
\usepackage{mathtools}
\usepackage{hyperref}
\usepackage{url}
\usepackage{algorithm}
\usepackage[noend]{algpseudocode}
%\documentstyle[nips14submit_09,times,art10]{article} % For LaTeX 2.09

\usepackage{mathrsfs}
\usepackage{graphicx}
\usepackage{caption}
\usepackage{subcaption}

\def\eQb#1\eQe{\begin{eqnarray*}#1\end{eqnarray*}}
\def\eQnb#1\eQne{\begin{eqnarray}#1\end{eqnarray}}
\providecommand{\e}[1]{\ensuremath{\times 10^{#1}}}
\providecommand{\pb}[0]{\pagebreak}


\def\Qb#1\Qe{\begin{question}#1\end{question}}
\def\Sb#1\Se{\begin{solution}#1\end{solution}}

\newenvironment{claim}[1]{\par\noindent\underline{Claim:}\space#1}{}
\newtheoremstyle{quest}{\topsep}{\topsep}{}{}{\bfseries}{}{ }{\thmname{#1}\thmnote{ #3}.}
\theoremstyle{quest}
\newtheorem*{definition}{Definition}
\newtheorem*{theorem}{Theorem}
\newtheorem*{lemma}{Lemma}
\newtheorem*{question}{Question}
\newtheorem*{preposition}{Preposition}
\newtheorem*{exercise}{Exercise}
\newtheorem*{challengeproblem}{Challenge Problem}
\newtheorem*{solution}{Solution}
\newtheorem*{remark}{Remark}
\usepackage{verbatimbox}
\usepackage{listings}

\title{Real Variables: \\
Problem Set V}


\author{
Youngduck Choi \\
Courant Institute of Mathematical Sciences \\
New York University \\
\texttt{yc1104@nyu.edu} \\
}


% The \author macro works with any number of authors. There are two commands
% used to separate the names and addresses of multiple authors: \And and \AND.
%
% Using \And between authors leaves it to \LaTeX{} to determine where to break
% the lines. Using \AND forces a linebreak at that point. So, if \LaTeX{}
% puts 3 of 4 authors names on the first line, and the last on the second
% line, try using \AND instead of \And before the third author name.

\newcommand{\fix}{\marginpar{FIX}}
\newcommand{\new}{\marginpar{NEW}}

\nipsfinalcopy % Uncomment for camera-ready version

\begin{document}


\maketitle

\begin{abstract}
This work contains solutions to the problem set 
V of Real Variables 2015 at NYU.
\end{abstract}

\section{Solutions}

\begin{question}[6.33]
\end{question}
\begin{solution}
Let $\{ f_n \}$ be a sequence of real-valued functions on $[a,b]$ 
that converges pointwise on $[a,b]$ to the real-valued function
$f$. We wish to show that $TV(f) \leq \liminf TV(f_n)$.
Fix $\epsilon > 0$. 
Let $P = \{x_0, ..., x_m\}$ be a partition of $[a,b]$. 
By the triangle inequality, it follows that
\eQb
V(f,P) &=& \sum_{k=0}^{m-1} |f(x_{k+1}) - f(x_k)| \\ 
&=& \sum_{k=0}^{m-1} |f(x_{k+1}) + f_n(x_{k+1}) - f_n(x_{k+1})
- f(x_k) + f_n(x_k) - f_n(x_k)| \\
&\leq& \sum_{k=0}^{m-1} |f(x_{k+1}) - f_n(x_{k+1})| + |f_n(x_{k+1})
- f_n(x_k)| + |f(x_k) - f_n(x_k)| \\ 
&\leq& V(f_n,P) + \sum_{k=1}^{m} |f(x_{k}) - f_n(x_{k})| + 
\sum_{k=0}^{m-1}|f(x_k) - f_n(x_k)|, \\ 
\eQe
for any $n$. Define $N = \max(N_0,...,N_k)$, where $N_i (0 \leq i 
\leq k)$ corresponds
to the convergence index for $\dfrac{\epsilon}{2m}$ at $x_i$. Then,
it follows that
\eQb
V(f,P) - \epsilon
&\leq&
V(f_n, P) 
\eQe 
for $n \geq N$. As $p$ was arbitrary, we can take the supremum
over $p$
on both sides, and obtain
\eQb
TV(f) - \epsilon &\leq& TV(f_n),
\eQe
for $n \geq N$.
As $\epsilon$ was arbitrary, we obtain that
\eQb
TV(f) &\leq& \inf_{n \geq N} TV(f_n).
\eQe
Now as $N \to \infty$, by the linearity of limit,
\eQb
TV(f,p) &\leq& \liminf_{N \to \infty} TV(f_n,p),
\eQe
as desired. $\qed$
\end{solution}

\bigskip

\begin{question}[4. Royden 6.42]
\end{question}
\begin{solution}
Let $f$ and $g$ be real-valued functions, that are 
absolutely continuous functions on $[a,b]$. 
We wish to show that $f+g$ is absolutely continuous on $[a,b]$.
Fix $\epsilon > 0$.
As $f$ and $g$ are both absolutely continuous on $[a,b]$, 
there exist $\delta_f , \delta_g > 0$, such that for any finite disjoint
open intervals $\{ (a_k, b_k) \}_{k=1}^{n}$ in $(a,b)$, 
\eQb
\sum_{k=1}^{n} [b_k - a_k] < \delta_f \implies 
\sum_{k=1}^{n} |f(b_k) - f(a_k)| < \dfrac{\epsilon}{2} \\
\sum_{k=1}^{n} [b_k - a_k] < \delta_g \implies
\sum_{k=1}^{n} |g_(b_k) - g(a_k)| < \dfrac{\epsilon}{2}. \\
\eQe
Define $\delta = \min(\delta_f, \delta_g)$. 
Let $\{(a_k, b_k) \}_{k=1}^{n}$
be a finite disjoint open intervals in $(a,b)$,
such that $\sum_{k=1}^{n} [b_k - a_k] < \delta$. 
It follows that
\eQb
\sum_{k=1}^{n} |f+g(b_k) - f+g(a_k)| 
&\leq& \sum_{k=1}^{n} |f(b_k) - f(a_k)| + |g(b_k) - g(a_k)| \\
&<& \dfrac{\epsilon}{2} + \dfrac{\epsilon}{2} = \epsilon .
\eQe 
Since $\epsilon$ was arbitrary,
we have shown that $f+g$ is absolutely continuous on $[a,b]$.

\bigskip

Let $f$ be a real-valued function, that is absolutely 
continuous on $[a,b]$. 
We show that $cf$, for any $c \in \mathbb{R}$, is absolutely continuous on 
$[a,b]$.
Let $c = 0$. Then $cf = 0$, which can 
trivially be shown to be absolutely continuous, 
as $f(c) = 0$ for any $c \in [a,b]$. Suppose $c \neq 0$.
As $f$ is absolutely continuous on $[a,b]$, there exists
$\delta_f > 0$, such that for any finite disjoint
open intervals $\{ (a_k, b_k) \}_{k=1}^{n}$ in $(a,b)$, 
\eQb
\sum_{k=1}^{n} |b_k - a_k| < \delta_f \implies
\sum_{k=1}^{n} |f(b_k) - f(a_k)| < \dfrac{\epsilon}{|c|}.
\eQe 
Define $\delta = \delta_f$. Let 
$\{(a_k, b_k) \}_{k=1}^{n}$ be a finite disjoint open intervals in
$(a,b)$ such that $\sum_{k=1}^{n} [b_k -
a_k] < \delta$.
It follows that
\eQb
\sum_{k=1}^{n} |cf(b_k) - cf(a_k)|
&=& |c|\sum_{k=1}^{n} |f(b_k) - f(a_k)| \\
&<& |c|\dfrac{\epsilon}{|c|} = \epsilon.
\eQe
Since $\epsilon$ was arbitrary,
combined with the $c=0$ case,
we have shown that $cf$, for any $c \in \mathbb{R}$, is absolutely continuous
on $[a,b]$.

\bigskip

Let $f$ be a real-valued function, that is absolutely continuous on $[a,b]$.
We wish to show that $f^2$ is absolutely continuous on $[a,b]$. As $f$ is absolutely
continuous, $f$ is continuous on $[a,b]$. Hence, by the Extreme
Value Theorem, there exists $M$ such that $|f| \leq M  \text{ on} 
\>\> [a,b]$. Fix $\epsilon > 0$.
As $f$ is absolutely continuous on $[a,b]$, there exists
$\delta_f > 0$, such that for any finite disjoint open intervals
$\{ (a_k, b_k) \}_{k=1}^{n}$ in $(a,b)$, 
\eQb
\sum_{k=1}^{n} |b_k - a_k| < \delta_f \implies 
\sum_{k=1}^{n} |f(b_k) - f(a_k)| < \dfrac{\epsilon}{2M}.
\eQe 
Define $\delta = \delta_f$. Let $\{(a_k, b_k)\}_{k=1}^{\infty}$ be a finite
disjoint open intervals in $(a,b)$ such that $\sum_{k=1}^{n} [b_k -
a_k] < \delta$.
It follows that
\eQb
\sum_{k=1}^{n} |f^2(b_k) - f^2(a_k)|
&=& \sum_{k=1}^{n} |f(b_k) - f(a_k)||f(b_k) + f(a_k)| \\
&\leq& 2M \sum_{k=1}^{n} |f(b_k) - f(a_k)| \\
&<& 2M \dfrac{\epsilon}{2M} = \epsilon.
\eQe
Since $\epsilon$ was arbitrary,
we have shown that $f^2$ is absolutely continuous on $[a,b]$.

\bigskip

Let $f$ and $g$ be real-valued functions, that are
absolutely continuous on $[a,b]$.
We wish to show that $fg$ is absolutely continuous on $[a,b]$.
Observe that 
\eQb
(f+g)^2 &=& f^2 + g^2 - 2fg,
\eQe 
which simplifies to
\eQb
fg = -\dfrac{1}{2}((f+g)^2 + (-f^2) + (-g^2)).
\eQe
As we have previously shown that a square of an absolutely continuous
function is absolutely continuous, and a scalar multiple of an
absolutely continuous function is absolutely continuous, we can deduce
that $fg$ is absolutely continuous on $[a,b]$. This completes the proof. $\qed$

\end{solution}

\bigskip

\begin{question}[4. 6.45]
\end{question}
\begin{solution}
Let $f$ be a real-valued function, that is absolutely continuous on $\mathbb{R}$. 
Let $g$ be a real-valued function, that is absolutely continuous and strictly monotone
on $[a,b]$. Without the loss of generality, we assume that $g$ is strictly increasing.
We wish to show that $f\circ g$ is absolutely continuous on $[a,b]$. 
Fix $\epsilon >0$.
As $f$ is absolutely continous on $\mathbb{R}$, it is also 
absolutely continuous on $[g(a), g(b)]$, which is a non-degenerate closed interval,
as $g$ is strictly increasing. 
there exists $\delta_f$, such that for any finite disjoint open intervals
$\{ (a_k, b_k) \}_{n=1}^{\infty}$ in $(g(a), g(b))$,
\eQb
\sum_{k=1}^{n} [b_k - a_k] < \delta_f \implies
\sum_{k=1}^{n} |f(b_k) - f(a_k)| < \epsilon  \>\> (*).
\eQe
As $g$ is absolutely continuous,
there exists $\delta_g$, such that for any finite disjoint open intervals 
$\{ (a_k, b_k) \}_{n=1}^{\infty}$ in 
$(a,b)$,
\eQb
\sum_{k=1}^{n} [b_k - a_k] < \delta_g \implies
\sum_{k=1}^{n} |g(b_k) - g(a_k)| < \delta_f.
\eQe 
Define $\delta = \delta_g$. Let $\{(a_k, b_k) \}_{k=1}^{n}$ be a finite 
djsjoint open intervals in $(a,b)$ such that 
$\sum_{k=1}^{n} [b_k - a_k] < \delta_g$. It follows that
$\sum_{k=1}^{n} [g(b_k) - g(a_k)] < \delta_f$. As $g$ is strictly increasing,
we observe that $\{ (g(a_k), g(b_k)) \}_{k=1}^{n}$ form a finite disjoint open 
intervals in $(g(a), g(b))$. Therefore, from $(*)$ it follows that  
\eQb
\sum_{k=1}^{n} |f\circ g(b_k) - f\circ g(a_k) | 
&<& \epsilon. \\
\eQe
Since $\epsilon$ was arbitrary, we have shown that $f\circ g$ is absolutely continuous on
$[a,b]$. $\qed$
\end{solution}

\end{document}
