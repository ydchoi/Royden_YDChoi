\documentclass{article} % For LaTeX2e
\usepackage{nips14submit_e,times}
\usepackage{amsmath}
\usepackage{amsthm}
\usepackage{amssymb}
\usepackage{mathtools}
\usepackage{hyperref}
\usepackage{url}
\usepackage{algorithm}
\usepackage[noend]{algpseudocode}
%\documentstyle[nips14submit_09,times,art10]{article} % For LaTeX 2.09

\usepackage{mathrsfs}
\usepackage{graphicx}
\usepackage{caption}
\usepackage{subcaption}

\def\eQb#1\eQe{\begin{eqnarray*}#1\end{eqnarray*}}
\def\eQnb#1\eQne{\begin{eqnarray}#1\end{eqnarray}}
\providecommand{\e}[1]{\ensuremath{\times 10^{#1}}}
\providecommand{\pb}[0]{\pagebreak}


\def\Qb#1\Qe{\begin{question}#1\end{question}}
\def\Sb#1\Se{\begin{solution}#1\end{solution}}

\newenvironment{claim}[1]{\par\noindent\underline{Claim:}\space#1}{}
\newtheoremstyle{quest}{\topsep}{\topsep}{}{}{\bfseries}{}{ }{\thmname{#1}\thmnote{ #3}.}
\theoremstyle{quest}
\newtheorem*{definition}{Definition}
\newtheorem*{theorem}{Theorem}
\newtheorem*{lemma}{Lemma}
\newtheorem*{question}{Question}
\newtheorem*{preposition}{Preposition}
\newtheorem*{exercise}{Exercise}
\newtheorem*{challengeproblem}{Challenge Problem}
\newtheorem*{solution}{Solution}
\newtheorem*{remark}{Remark}
\usepackage{verbatimbox}
\usepackage{listings}

\title{Real Variables: \\
Problem Set V}


\author{
Youngduck Choi \\
Courant Institute of Mathematical Sciences \\
New York University \\
\texttt{yc1104@nyu.edu} \\
}


% The \author macro works with any number of authors. There are two commands
% used to separate the names and addresses of multiple authors: \And and \AND.
%
% Using \And between authors leaves it to \LaTeX{} to determine where to break
% the lines. Using \AND forces a linebreak at that point. So, if \LaTeX{}
% puts 3 of 4 authors names on the first line, and the last on the second
% line, try using \AND instead of \And before the third author name.

\newcommand{\fix}{\marginpar{FIX}}
\newcommand{\new}{\marginpar{NEW}}

\nipsfinalcopy % Uncomment for camera-ready version

\begin{document}


\maketitle

\begin{abstract}
This work contains solutions to the problem set 
V of Real Variables 2015 at NYU.
\end{abstract}

\section{Solutions}

\begin{question}[6.10]
\end{question}
\begin{solution}
Let $x_1, x_2 \in (a,b)$, such that $x_1 < x_2$. Then, we have
\eQb
f(x_1) &=& \sum_{k=1}^{\infty} l((c_k, d_k) \cap (-\infty, x_1)) \\
f(x_2) &=& \sum_{k=1}^{\infty} l((c_k, d_k) \cap (-\infty, x_2)).
\eQe
As $x_1 < x_2$, by the monotonicity property of the length function
, we have
\eQb
l((c_k, d_k) \cap (-\infty, x_1)) &\leq& 
l((c_k, d_k) \cap (-\infty, x_2)),
\eQe
for all $k \geq 1$. It follows that $f(x_1) \leq f(x_2)$. Hence,
$f$ is increasing. 
We show that $f$ fails to be differentiable at
each point in $E$, which is a set of measure zero contained in the open interval
$(a,b)$. Let $x \in E$. Then, by the preceding problem, there exist a countable
collection of open intervals contained in $(a,b)$, $\{ (c_k, d_k) \}_{k=1}^{\infty}$
such taht each point in $E$ belongs to infinitely many intervals in the collection
and $\sum_{k=1}^{\infty} d_k - c_k < \infty$. Let $\{ (c_{k_i}, d_{k_i}) \}_{i=1}^{\infty}$
be the sub-collection such that $x \in (c_{k_i}, d_{k_i})$ for all $i$. Then, there exist
a finite sub-cover $\{(c_{k_i}, d_{k_i}) \}_{i=1}^{n}$ that $x$ belongs to. Since, $n$ is finite,
as intersection of finite open sets is open, there exists $a_n$ such that 
\eQb
B(x,a_n) \in \cup_{k=1}^{n} (c_{k_i}, d_{k_i}),
\eQe
such that $(B,a_n)$ denotes the ball of radius $a_n$, centered at $x$. Observe that
\eQb
f(x + a_n) - f(x) &\geq& \sum_{i=1}^{n} l((c_{k_i}, d_{k_i}) \cap(x, x+a_n)) \\
&=& na_n.
\eQe
It follows that
\eQb
\bar{D}f(x) &=& \lim_{h \to 0} \sup_{0< |t| \leq h} \{ \dfrac{f(x+t) - f(x)}{t} \} \\
&=& \lim_{h \to 0} \sup_{0 < |t| \leq h} \dfrac{na_n}{a_n} \\
&\geq& n.
\eQe
Since $n$ was arbitrary, we have that 
\eQb
\bar{D}f(x) = \infty,
\eQe
which is not finite, and by definition, $x$ is not differentiable at $x$. Therefore,
$f$ fails to be differntiable at each point in $E$. $\qed$

\end{solution}

\bigskip

\begin{question}[6.33]
\end{question}
\begin{solution}
Let $\{ f_n \}$ be a sequence of real-valued functions on $[a,b]$ 
that converges pointwise on $[a,b]$ to the real-valued function
$f$. We wish to show that $TV(f) \leq \liminf TV(f_n)$.
Fix $P = \{x_0, ..., x_m\}$ be a partition of $[a,b]$. 
As $f_n \to f$ pointwise, we have
\eQb
V(f,P) &=& \sum_{k=0}^{m-1} |f(x_{k+1}) - f(x_k)| \\
&=& \lim_{n \to \infty} \sum_{k=0}^{m-1} |f_n(x_{k+1}) - f_n(x_k)| \\
&=& \lim_{n \to \infty} V(f_n, P). \\ 
\eQe
By the definition of total variation, it follows that
\eQb
V(f_n,P) &\leq& TV(f_n),
\eQe
for all $n$. Consequently, we obtain
\eQb
V(f,P) &\leq& \liminf_{n \to \infty} TV(f_n),
\eQe
and since $P$ was arbitrary, we finally have that 
\eQb
TV(f) &\leq& \liminf_{n \to \infty} TV(f_n),
\eQe
as desired. $\qed$
\end{solution}

\bigskip

\begin{question}[4. Royden 6.42]
\end{question}
\begin{solution}
Let $f$ and $g$ be real-valued functions, that are 
absolutely continuous functions on $[a,b]$. 
We wish to show that $f+g$ is absolutely continuous on $[a,b]$.
Fix $\epsilon > 0$.
As $f$ and $g$ are both absolutely continuous on $[a,b]$, 
there exist $\delta_f , \delta_g > 0$, such that for any finite disjoint
open intervals $\{ (a_k, b_k) \}_{k=1}^{n}$ in $(a,b)$, 
\eQb
\sum_{k=1}^{n} [b_k - a_k] < \delta_f \implies 
\sum_{k=1}^{n} |f(b_k) - f(a_k)| < \dfrac{\epsilon}{2} \\
\sum_{k=1}^{n} [b_k - a_k] < \delta_g \implies
\sum_{k=1}^{n} |g_(b_k) - g(a_k)| < \dfrac{\epsilon}{2}. \\
\eQe
Define $\delta = \min(\delta_f, \delta_g)$. 
Let $\{(a_k, b_k) \}_{k=1}^{n}$
be a finite disjoint open intervals in $(a,b)$,
such that $\sum_{k=1}^{n} [b_k - a_k] < \delta$. 
It follows that
\eQb
\sum_{k=1}^{n} |f+g(b_k) - f+g(a_k)| 
&\leq& \sum_{k=1}^{n} |f(b_k) - f(a_k)| + |g(b_k) - g(a_k)| \\
&<& \dfrac{\epsilon}{2} + \dfrac{\epsilon}{2} = \epsilon .
\eQe 
Since $\epsilon$ was arbitrary,
we have shown that $f+g$ is absolutely continuous on $[a,b]$.

\bigskip

Let $f$ be a real-valued function, that is absolutely 
continuous on $[a,b]$. 
We show that $cf$, for any $c \in \mathbb{R}$, is absolutely continuous on 
$[a,b]$.
Let $c = 0$. Then $cf = 0$, which can 
trivially be shown to be absolutely continuous, 
as $f(c) = 0$ for any $c \in [a,b]$. Suppose $c \neq 0$.
As $f$ is absolutely continuous on $[a,b]$, there exists
$\delta_f > 0$, such that for any finite disjoint
open intervals $\{ (a_k, b_k) \}_{k=1}^{n}$ in $(a,b)$, 
\eQb
\sum_{k=1}^{n} |b_k - a_k| < \delta_f \implies
\sum_{k=1}^{n} |f(b_k) - f(a_k)| < \dfrac{\epsilon}{|c|}.
\eQe 
Define $\delta = \delta_f$. Let 
$\{(a_k, b_k) \}_{k=1}^{n}$ be a finite disjoint open intervals in
$(a,b)$ such that $\sum_{k=1}^{n} [b_k -
a_k] < \delta$.
It follows that
\eQb
\sum_{k=1}^{n} |cf(b_k) - cf(a_k)|
&=& |c|\sum_{k=1}^{n} |f(b_k) - f(a_k)| \\
&<& |c|\dfrac{\epsilon}{|c|} = \epsilon.
\eQe
Since $\epsilon$ was arbitrary,
combined with the $c=0$ case,
we have shown that $cf$, for any $c \in \mathbb{R}$, is absolutely continuous
on $[a,b]$.

\bigskip

Let $f$ be a real-valued function, that is absolutely continuous on $[a,b]$.
We wish to show that $f^2$ is absolutely continuous on $[a,b]$. As $f$ is absolutely
continuous, $f$ is continuous on $[a,b]$. Hence, by the Extreme
Value Theorem, there exists $M$ such that $|f| \leq M  \text{ on} 
\>\> [a,b]$. Fix $\epsilon > 0$.
As $f$ is absolutely continuous on $[a,b]$, there exists
$\delta_f > 0$, such that for any finite disjoint open intervals
$\{ (a_k, b_k) \}_{k=1}^{n}$ in $(a,b)$, 
\eQb
\sum_{k=1}^{n} |b_k - a_k| < \delta_f \implies 
\sum_{k=1}^{n} |f(b_k) - f(a_k)| < \dfrac{\epsilon}{2M}.
\eQe 
Define $\delta = \delta_f$. Let $\{(a_k, b_k)\}_{k=1}^{\infty}$ be a finite
disjoint open intervals in $(a,b)$ such that $\sum_{k=1}^{n} [b_k -
a_k] < \delta$.
It follows that
\eQb
\sum_{k=1}^{n} |f^2(b_k) - f^2(a_k)|
&=& \sum_{k=1}^{n} |f(b_k) - f(a_k)||f(b_k) + f(a_k)| \\
&\leq& 2M \sum_{k=1}^{n} |f(b_k) - f(a_k)| \\
&<& 2M \dfrac{\epsilon}{2M} = \epsilon.
\eQe
Since $\epsilon$ was arbitrary,
we have shown that $f^2$ is absolutely continuous on $[a,b]$.

\bigskip

Let $f$ and $g$ be real-valued functions, that are
absolutely continuous on $[a,b]$.
We wish to show that $fg$ is absolutely continuous on $[a,b]$.
Observe that 
\eQb
(f+g)^2 &=& f^2 + g^2 - 2fg,
\eQe 
which simplifies to
\eQb
fg = -\dfrac{1}{2}((f+g)^2 + (-f^2) + (-g^2)).
\eQe
As we have previously shown that a square of an absolutely continuous
function is absolutely continuous, and a scalar multiple of an
absolutely continuous function is absolutely continuous, we can deduce
that $fg$ is absolutely continuous on $[a,b]$. This completes the proof. $\qed$

\end{solution}

\bigskip

\begin{question}[4. 6.45]
\end{question}
\begin{solution}
Let $f$ be a real-valued function, that is absolutely continuous on $\mathbb{R}$. 
Let $g$ be a real-valued function, that is absolutely continuous and strictly monotone
on $[a,b]$. Without the loss of generality, we assume that $g$ is strictly increasing.
We wish to show that $f\circ g$ is absolutely continuous on $[a,b]$. 
Fix $\epsilon >0$.
As $f$ is absolutely continous on $\mathbb{R}$, it is also 
absolutely continuous on $[g(a), g(b)]$, which is a non-degenerate closed interval,
as $g$ is strictly increasing. 
there exists $\delta_f$, such that for any finite disjoint open intervals
$\{ (a_k, b_k) \}_{n=1}^{\infty}$ in $(g(a), g(b))$,
\eQb
\sum_{k=1}^{n} [b_k - a_k] < \delta_f \implies
\sum_{k=1}^{n} |f(b_k) - f(a_k)| < \epsilon  \>\> (*).
\eQe
As $g$ is absolutely continuous,
there exists $\delta_g$, such that for any finite disjoint open intervals 
$\{ (a_k, b_k) \}_{n=1}^{\infty}$ in 
$(a,b)$,
\eQb
\sum_{k=1}^{n} [b_k - a_k] < \delta_g \implies
\sum_{k=1}^{n} |g(b_k) - g(a_k)| < \delta_f.
\eQe 
Define $\delta = \delta_g$. Let $\{(a_k, b_k) \}_{k=1}^{n}$ be a finite 
djsjoint open intervals in $(a,b)$ such that 
$\sum_{k=1}^{n} [b_k - a_k] < \delta_g$. It follows that
$\sum_{k=1}^{n} [g(b_k) - g(a_k)] < \delta_f$. As $g$ is strictly increasing,
we observe that $\{ (g(a_k), g(b_k)) \}_{k=1}^{n}$ form a finite disjoint open 
intervals in $(g(a), g(b))$. Therefore, from $(*)$ it follows that  
\eQb
\sum_{k=1}^{n} |f\circ g(b_k) - f\circ g(a_k) | 
&<& \epsilon. \\
\eQe
Since $\epsilon$ was arbitrary, we have shown that $f\circ g$ is absolutely continuous on
$[a,b]$. $\qed$
\end{solution}

\bigskip

\begin{question}[6.55]
\end{question}
\begin{solution}
\textbf{(ii)} Assume that $f$ is absolutely continuous. Let $P =
\{x_0, ..., x_k \}$. Then, by the additivity over domain of integration,
and the absolute continuity of $f$, we have
\eQb
\int_{a}^{b} |f'(x)|dx &\geq& \sum_{i=0}^{k-1} 
|\int_{x_i}^{x_{i+1}} f'(x) | dx \\
&=& V(f,P).
\eQe
Since $P$ was arbitrary, we have $\int_{a}^{b}|f'| \geq TV(f)$. Hence,
with $\int_{a}^{b}|f'| \leq TV(f)$, we can conclude that
\eQb
\int_{a}^{b}|f'| = TV(f). 
\eQe

\end{solution}



\begin{question}[6.56]
\end{question}
\begin{solution}
Let $g$ be strictly increasing an absolutely continuous on $[a,b]$. \\

\smallskip

\textbf{(i)} Let $O$ be an open subset of $(a,b)$. Then, $O$ can be represented as
a countable union of disjoint intervals in $(a,b)$: 
\eQb
O &=& \cup_{k=1}^{\infty} (a_k,b_k),
\eQe
and since $g$ is strictly increasing, we have
$\{(g(a_k),g(b_k))\}_{k=1}^{\infty}$ forms a collection
of disjoint intervals, and
\eQb
g(O) &=& \cup_{k=1}^{\infty} (g(a_k),g(b_k)). \\
\eQe
Therefore, by the countable additivity of measure, it follows that
\eQb
m(g(O)) &=& \sum_{k=1}^{\infty} g(b_k) - g(a_k).
\eQe
On the other hand, by the countable additivity of integration, and as $g$ is absolutely continuous,
we have
\eQb
\int_{O} g'(x)dx &=& \sum_{k=1}^{\infty} \int_{g(a,k)}^{g(b_k)} g'(x)dx \\
&=& \sum_{k=1}^{\infty} g(b_k) - g(a_k).
\eQe
Therefore, $m(g(O)) = \int_{O}g'(x)dx$, as desired.

\smallskip

\textbf{(ii)}

\smallskip

\textbf{(iii)}
Let $E$ be a meazuer zero set.
We have previously shown that a continuous map carries a measure zero set to a measure zero set. Hence,
$g(E)$ has measure zero. Furthermore, we have that an integral over a measure zero set is zero.
Therfore, we have that $m(g(E)) = 0 = \int_{E} g'(x) dx$ as desired.

\smallskip

\textbf{(iv)}
Let $A$ be any measurable set of $[a,b]$.
By the outer approximation of a measurable set via $G-\delta$ set, there exists $G-\delta$ set
$G$ such that $m(G\setminus A) = 0$ and $A \subseteq G$. Then, by the finite additivity of measure
we have
\eQb
m(g(A)) &=& m(g(G\setminus A \cup A)) \\
&=& m(g(G\setminus A)) + m(g(A)) \\
&=& m(g(G)).
\eQe
On the other hand, by the additivity over domain property
of integration and the fact that any integral on
a measure zero set is zero, we have
\eQb
\int_{A} g'(x)dx &=& \int_{A} g'(x) dx + \int_{G\setminus A} g'(x)dx \\
&=& \int_{G} g'(x)dx. 
\eQe
By the preceding result with $G-\delta$ sets, $LHS = RHS$. Hence,
we have $m(g(A)) = \int_{A}g'(x)dx$ for any measurable set of $[a,b]$.

\smallskip

\textbf{(v)}
Let $c = g(a)$ and $d = g(b)$. We can write the simple function $\psi$ as
\eQb
\psi &=& \sum_{k=1}^{n} c_k \chi_{E_k}.
\eQe
Then, by the countable additivity over domain property of integration,
we have
\eQb
\int_{c}^{d} \psi(y)dy &=& \int_{c}^{d} \sum_{k=1}^{n} c_k \chi_{E_k}(y)dy \\
&=& \sum_{k=1}^{n} c_k \int_{c}^{d} \chi_{E_k}(y) dy \\
&=& \sum_{k=1}^{n} c_k m(E_k).
\eQe
Similarly, the RHS can be computed to be the same sum as desired.
\smallskip

\textbf{(vi)}
Since $f$ is non-negative integrable function on $[c,d]$, there 
exists a sequence of increasing simple functions $\{\phi_n\}$ such that
$\phi_n \to f$ pointwise and $|\phi_n| \leq |f|$ on $[c,d]$ for all $n$.
As $f$ is integrable and dominates $\phi_n$ for all $n$, by the Dominated
Convergence Theorem, we have 
\eQb
\lim_{n \to \infty} \int_{c}^{d} \phi_n(y)dy &=& 
\int_{c}^{d} f(y)dy.
\eQe
As $g$ is strictly increasing and absolutely continuous, we have that
$g'$ is non-negative and integrable with $\int_{c}^{d}g'(x) dx = g(d) - g(c)
< \infty$. Therefore, we have that $\phi_{n}(g)g'$ and $f(g)g'$ are
both integrable and non-negative, and $\phi_{n}(g)g' \to
f(g)g'$ pointwise and $f(g)g'$ dominates $\phi_{n}(g)g'$ for all $n$. 
Therefore, by the Dominated Convergence Theorem, we have 
\eQb
\lim_{n \to \infty} \int_{c}^{d} \phi_{n}(g(x))g'(x) dx&=& 
\int_{c}^{d} f(g(x))g'(x) dx. 
\eQe 
Since $\int_{c}^{d} \phi_n(y)dy = \int_{c}^{d} \phi_n(g(x))g'(x)dx$
for all $n$ by the previous result, we have that
\eQb
\int_{c}^{d} f(y)dy &=& \int_{c}^{d} f(g(x))g'(x)dx,
\eQe
as desired.

\smallskip

\textbf{(vii)}
As we have $(vi)$, by setting $f = \chi_{O}$, we have
\eQb
\int_{c}^{d} \chi_{O}(y) dy &=& 
\int_{a}^{b} \chi_{O}(g(x))g'(x) dx .
\eQe
As $\chi_{O}$ is a characteristic function, we have
\eQb
\int_{c}^{d} \chi_{O}(y)dy &=& m(O) \\
\int_{a}^{b} \chi_{O}(g(x)) g'(x) dx &=& \int_{O} g'(x)dx.
\eQe
Hence, $(i)$ holds as desired. $\qed$



\end{solution}


\end{document}
