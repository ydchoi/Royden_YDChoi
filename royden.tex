\documentclass{article} % For LaTeX2e
\usepackage{nips14submit_e,times}
\usepackage{amsmath}
\usepackage{amsthm}
\usepackage{amssymb}
\usepackage{mathtools}
\usepackage{hyperref}
\usepackage{url}
\usepackage{algorithm}
\usepackage[noend]{algpseudocode}
%\documentstyle[nips14submit_09,times,art10]{article} % For LaTeX 2.09

\usepackage{graphicx}
\usepackage{caption}
\usepackage{subcaption}

\def\eQb#1\eQe{\begin{eqnarray*}#1\end{eqnarray*}}
\def\eQnb#1\eQne{\begin{eqnarray}#1\end{eqnarray}}
\providecommand{\e}[1]{\ensuremath{\times 10^{#1}}}
\providecommand{\pb}[0]{\pagebreak}


\def\Qb#1\Qe{\begin{question}#1\end{question}}
\def\Sb#1\Se{\begin{solution}#1\end{solution}}

\newenvironment{claim}[1]{\par\noindent\underline{Claim:}\space#1}{}
\newtheoremstyle{quest}{\topsep}{\topsep}{}{}{\bfseries}{}{ }{\thmname{#1}\thmnote{ #3}.}
\theoremstyle{quest}
\newtheorem*{definition}{Definition}
\newtheorem*{theorem}{Theorem}
\newtheorem*{question}{Question}
\newtheorem*{exercise}{Exercise}
\newtheorem*{challengeproblem}{Challenge Problem}
\newtheorem*{solution}{Solution}
\newtheorem*{remark}{Remark}
\usepackage{verbatimbox}
\usepackage{listings}
\title{Royden}


\author{
Youngduck Choi \\
Courant Institute of Mathematical Sciences \\
New York University \\
\texttt{yc1104@nyu.edu} \\
}


% The \author macro works with any number of authors. There are two commands
% used to separate the names and addresses of multiple authors: \And and \AND.
%
% Using \And between authors leaves it to \LaTeX{} to determine where to break
% the lines. Using \AND forces a linebreak at that point. So, if \LaTeX{}
% puts 3 of 4 authors names on the first line, and the last on the second
% line, try using \AND instead of \And before the third author name.

\newcommand{\fix}{\marginpar{FIX}}
\newcommand{\new}{\marginpar{NEW}}

\nipsfinalcopy % Uncomment for camera-ready version

\begin{document}


\maketitle

\begin{abstract}
This work contains the solutions to Royden's Real Variables.
\end{abstract}

\section{Solutions}

\Qb[1. Royden 1.1-1 (Distributive Property of Multiplicative Inverse in Reals)]
\Qe
\Sb
Assume that $a \neq 0$ and $b \neq 0$. From the multiplicative identity axiom,
we have that a multiplicative inverse exists for $a$ and $b$ individually, which we denote as
$a^{-1}$ and $b^{-1}$ respectively.
Now, consider the expression $(ab)(a^{-1}b^{-1})$, where $ab$ denotes the 
product of $a$ and $b$, and $a^{-1}b^{-1}$ denotes the product of $a^{-1}$ and $b^{-1}$.
From the commutativity of multiplication,
we obtain
\eQb
(ab)(a^{-1}b^{-1}) = (ab)(b^{-1}a^{-1}).
\eQe
Using the associativity of multiplication and iteratively substituting $bb^{-1} = 1$ and $aa^{-1} = 1$,
we have
\eQb
(ab)(a^{-1}b^{-1}) = 1,
\eQe
where $1$ denotes the identity as usual. Hence, the product, $a^{-1}b^{-1}$ satisfies definition
of multiplicative inverse with respect to the $ab$ term whose multiplicative inverse can be denoted
as $(ab)^{-1}$ by convention. Therefore, we obtain that
\eQb
(ab)^{-1} = a^{-1}b^{-1},
\eQe
as desired.
\Se

\bigskip

\Qb[Royden 1.1-3]
\Qe
\begin{solution}
Let $E$ be a nonemepty set of real numbers.\\ 
$\mathbf{( \Leftarrow )}$ Assume that 
$E$ consists of a single point, which we denote as $x$. We claim that
$\text{inf}E = x$ and $\text{sup}E = x$. As we have $x \leq x$, we see that $x$ is an upper bound
for $E$. Suppose that there exists an upper bound for $E$, $a$, that is smaller than $x$, namely 
$a < x$. This is a contradiction to the fact that $a$ is an upper bound as it is required to have
$x \leq a$ with $x \in E$. Hence, there does not exists any upper bound for $E$ that is smaller than
$x$. By definition of supremum, we have that $\text{sup}E = x$. By symmtry, we can see that
$\text{inf}E = x$ as well. Therefore, $\text{inf}E = \text{sup}E$. \\ 

\smallskip

$\mathbf{( \Rightarrow )}$ Assume that $\text{inf} E = \text{sup} E$. 
Given the assumption, let us denote the infimum and supremum for $E$ as a
single real number $a$. Then, by definition of infimum, any $x$ in $E$, we have
$a \leq x$. Furthermore, by definition of supremum, any $x$ in $E$, we have $x \leq a$.
The only real number that can satisfy the two given equality is $a$ itself. We also know
that $a$ must be in $E$ as $E$ is a nonempty set of reals. Therefore, 
we have shown that $E = \{ a \}$, and that $E$ consits of a single point.
\end{solution}

\pagebreak
\end{document}
