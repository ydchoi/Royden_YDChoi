\documentclass{article} % For LaTeX2e
\usepackage{nips14submit_e,times}
\usepackage{amsmath}
\usepackage{amsthm}
\usepackage{amssymb}
\usepackage{mathtools}
\usepackage{hyperref}
\usepackage{url}
\usepackage{algorithm}
\usepackage[noend]{algpseudocode}
%\documentstyle[nips14submit_09,times,art10]{article} % For LaTeX 2.09

\usepackage{graphicx}
\usepackage{caption}
\usepackage{subcaption}

\def\eQb#1\eQe{\begin{eqnarray*}#1\end{eqnarray*}}
\def\eQnb#1\eQne{\begin{eqnarray}#1\end{eqnarray}}
\providecommand{\e}[1]{\ensuremath{\times 10^{#1}}}
\providecommand{\pb}[0]{\pagebreak}


\def\Qb#1\Qe{\begin{question}#1\end{question}}
\def\Sb#1\Se{\begin{solution}#1\end{solution}}

\newenvironment{claim}[1]{\par\noindent\underline{Claim:}\space#1}{}
\newtheoremstyle{quest}{\topsep}{\topsep}{}{}{\bfseries}{}{ }{\thmname{#1}\thmnote{ #3}.}
\theoremstyle{quest}
\newtheorem*{definition}{Definition}
\newtheorem*{theorem}{Theorem}
\newtheorem*{lemma}{Lemma}
\newtheorem*{question}{Question}
\newtheorem*{preposition}{Preposition}
\newtheorem*{exercise}{Exercise}
\newtheorem*{challengeproblem}{Challenge Problem}
\newtheorem*{solution}{Solution}
\newtheorem*{remark}{Remark}
\usepackage{verbatimbox}
\usepackage{listings}
\title{Royden}


\author{
Youngduck Choi \\
Courant Institute of Mathematical Sciences \\
New York University \\
\texttt{yc1104@nyu.edu} \\
}


% The \author macro works with any number of authors. There are two commands
% used to separate the names and addresses of multiple authors: \And and \AND.
%
% Using \And between authors leaves it to \LaTeX{} to determine where to break
% the lines. Using \AND forces a linebreak at that point. So, if \LaTeX{}
% puts 3 of 4 authors names on the first line, and the last on the second
% line, try using \AND instead of \And before the third author name.

\newcommand{\fix}{\marginpar{FIX}}
\newcommand{\new}{\marginpar{NEW}}

\nipsfinalcopy % Uncomment for camera-ready version

\begin{document}


\maketitle

\begin{abstract}
This work contains the solutions to Royden's Real Variables.
\end{abstract}

\section{Chapter III}
\begin{theorem}[3. Egoroff's Theorem]
\end{theorem}

\section{Chapter II}
\Qb[Royden 2.1-1]
\Qe
\begin{solution}
Let $m$ be a set function defined for all sets in a $\sigma-$algebra $\mathcal{A}$ 
with values in $[0, \infty]$. Assume that $m$ is countably additive over countable 
disjoint collections of sets in $\mathcal{A}$. Furthermore, 
assume that $A$ and $B$ are two sets in
$\mathcal{A}$ with $A \subseteq B$. 
Given that $m$ is countably additive over countable disjoint collections of sets in
$\mathcal{A}$, we have 
\eQb
m(B) &=& m(A) + m(B \setminus A),
\eQe
where $B \setminus A$ is a well-defined set with $A \subseteq B$ assumption, thus
$A$ and $B \setminus A$ forming a valid countable disjoint collections of sets 
whose union is $B$.
With $m$ being
a set function with values in $[0, \infty]$, we obtain
$m(B) = m(A) + r$,
where $r$ denotes some non-negative real value. Therefore, we finally get
\eQb
m(A) \leq m(B).
\eQe
Hence, we have shown that the given set function $m$ has the monotonicity 
property.
\end{solution}

\bigskip

\Qb[Royden 2.1-2]
\Qe
\begin{solution}

Let $m$ be a set function defined for all sets in a $\sigma-$algebra $\mathcal{A}$ 
with values in $[0, \infty]$. Assume that $m$ is countably additive over countable 
disjoint collections of sets in $\mathcal{A}$. Furthermore, assume that there exists 
a set $A$ in the collection $\mathcal{A}$ such that $m(A) < \infty$. Using the countably
additive property with a collection $\{ A, \emptyset \}$, we obtain
\eQb
m(A \cup \emptyset ) &=& m(A) + m(\emptyset).
\eQe
Substituting $A \cup \emptyset = A$ and subtracting $m(A)$ from both sides, granted with finiteness of
$m(A)$, we get
\eQb
m(\emptyset ) &=& 0,
\eQe
as desired. Hence, we have shown that if there is a set $A$ in the collection $\mathcal{A}$ for
whcih $m(A) < \infty$, then $m( \emptyset ) = 0$.
\end{solution}

\bigskip

\Qb[Royden 2.1-3]
\Qe
\begin{solution}

\end{solution}

\bigskip

\Qb[Royden 2.1-5 (A Countable Set Has Outer Measure Zero)]
\Qe
\begin{solution}
We know that any countable set has outer measure zero.
Using the fact that the outer measure of an interval is its length yields $m^{*}( [0,1] ) = 1$.
Therefore, $[0,1]$ cannot countable.
\end{solution}

\bigskip

\Qb[: Royden 2.1-6]
\Qe
\begin{solution}
Let $Q$ and $A$ denote the set of rationals and irrationals in the interval $[0,1]$
respectively. Consider a countable collection of sets $\{ Q, A \}$.
Since outer measure is countably subadditive, we have 
\eQb
m^{*}( Q \cup A) & \leq & m^*(Q) + m^{*}(A).
\eQe
As $Q$ is a countable set whose outer measure is zero 
and $Q \cup A = [0,1]$ by construction, we obtain
\eQb
m^{*}([0,1]) & \leq & m^{*}(A).
\eQe
As the outer measure of an interval is its length, we have
\eQb
1 & \leq & m^{*}(A).
\eQe
Using the monotonicity property of outer measure with $A \subset [0,1]$,
we also see
\eQb
m^{*}(A) & \leq & 1,
\eQe
thereby showing that $m^{*}(A) = 1$. 
\end{solution}

\bigskip

\Qb[: Royden 2.3-Proposition 4] 
\Qe
\begin{solution}
We want to show that any set of outer measure zero is measurable, which further implies that
any countable set is measurable. Let the set $E$ to have outer measure zero, $m^{*}(E) = 0$.
Let $A$ be any set. Since
\eQb
A \cap E \subseteq E \> \text{ and } \> A \cap E^c \subseteq A,
\eQe
by the monotonicity of outer measure, we obtain
\eQb
m^{*}(A \cap E) \leq m^{*}(E) = 0 \> \text{ and } \> m^{*}(A \cap E^c) \leq m^{*}(A).
\eQe
It is important to note that it suffices to show the above statement to show that the set $E$
is measurable, as the inequality,
\eQb
m^{*} \leq m^{*}(A \cap E) + m^{*}( A \cap E^c),
\eQe
trivially holds with the finite subadditive property of outer measure.
\end{solution}

\bigskip

\begin{definition}
A collection of subsets of $\mathbb{R}$ is an algebra, provided that it 
contains $\mathbb{R}$ and is closed with respect to the formation of complements 
and finite unions. A collection of subsets of $\mathbb{R}$ is an $\sigma$-algebra,
provided that it contains $\mathbb{R}$ and is closed with respect to the formation
of complements and countable unions.
\end{definition}

\bigskip

\begin{preposition}
Algebra is closed with respect to the formation of finite intersection.
\end{preposition}
\begin{proof}
Let $\mathcal{A}$ be an algebra and $\{ A_k \}_{k=1}^{n}$ be a finite collection of
sets in $\mathcal{A}$. We want to show that $\cap_{k=1}^{n} A_k$ is in $\mathcal{A}$.
Since algebra is closed with respect to the formation of finite union
and complements, we have 
\eQb
\cup_{k=1}^{n} A_k^{C} \> \text{ is in} \> \mathcal{A}. 
\eQe
By applying De Morgan's identity and closedness of complements iteratively, we see that
\eQb
{(\cap_{k=1}^{n} A_k)}^{C} \> \text{ is in} \> \mathcal{A} \>\> \text{ and } \>\>
\cap_{k=1}^{n} A_k \> \text{ is in} \> \mathcal{A}, 
\eQe
thereby showing that an algebra is closed with respect to the formation of finite intersection.
\end{proof}

\bigskip

\begin{preposition}
The collection of measurable sets is an algebra. The union of a countable collection of measurable sets
is also the union of a countable disjoint collection of measurable sets. 
\end{preposition}
\begin{proof} Since we know that the union of a finite collection of measurable sets is measurable 
and the complement of a measurable set is measurable, we immediately see that the collection of
measurable sets is an algebra. Now, consider the union of a countable collection of measurable sets,
denoted as $\cup_{i=1}^{\infty}E_i$. Define a new countable collection of sets 
$\{ A_i \}_{i=1}^{\infty} $
by
\eQb
A_i \triangleq E_i \sim \cup_{k \neq i}^{\infty} E_k , 
\eQe
where each $A_i$ is disjoint and 
indeed in the measurable sets as the construction of it respects the algebra.
Furthermore, the union of $\{ A_i \}$ collection is precisely 
the union of $E_i$ as well. Therefore, we have 
proven the preposition.
\end{proof}

\bigskip

\begin{preposition}
The union of a countable collection of measurable sets is measurable. Thus,
the collection of measurable sets is an $\sigma$-algebra.
\end{preposition}
\begin{proof}
\end{proof}

\bigskip

\Qb[2.6-29: Equivalence relation]
\Qe
\begin{solution}
\textbf{(i)} We want to show that the rational equivalence defines an equivalence 
relation on any set. Let $A$ be a set and $R$ be a relation set induced by 
the rational equivalence. For $x \in A$, $x - x = 0$ is clearly a rational.
Hence, the relation is reflexive. For $(x,y) \in R$, we have that $x-y$ is a rational.
As a negation of a rational is a rational, $(y,x) \in R$. Thus, the relation is symmetric.
For $(x,y), (y,z) \in R$, we have that $x-y$ and $y-z$ are rationals. Notice that
negating $y-z$ term and subtracting it from $x-y$ gives $x-z$. Since rationals are closed
under subtraction, we have that $x-z$ is a rational and $(x,z) \in R$. Therefore, the relation
is transitive, thereby showing that the rational equivalence defines an equivalence relation 
on any set. \\

\smallskip

\textbf{(ii)} Irrationals  

\smallskip

\textbf{(iii)} The irrationally equivalent as a relation fails to be reflexive on both 
$\mathbb{R}$ and $\mathbb{Q}$, as $0$ is not a irrational number.

\end{solution}


\section{Chapter I}

\Qb[: Royden 1.1-1 (Distributive Property of Multiplicative Inverse in Reals)]
\Qe
\Sb
Assume that $a \neq 0$ and $b \neq 0$. From the multiplicative identity axiom,
we have that a multiplicative inverse exists for $a$ and $b$ individually, which we denote as
$a^{-1}$ and $b^{-1}$ respectively.
Now, consider the expression $(ab)(a^{-1}b^{-1})$, where $ab$ denotes the 
product of $a$ and $b$, and $a^{-1}b^{-1}$ denotes the product of $a^{-1}$ and $b^{-1}$.
From the commutativity of multiplication,
we obtain
\eQb
(ab)(a^{-1}b^{-1}) = (ab)(b^{-1}a^{-1}).
\eQe
Using the associativity of multiplication and iteratively substituting $bb^{-1} = 1$ and $aa^{-1} = 1$,
we have
\eQb
(ab)(a^{-1}b^{-1}) = 1,
\eQe
where $1$ denotes the identity as usual. Hence, the product, $a^{-1}b^{-1}$ satisfies definition
of multiplicative inverse with respect to the $ab$ term whose multiplicative inverse can be denoted
as $(ab)^{-1}$ by convention. Therefore, we obtain that
\eQb
(ab)^{-1} = a^{-1}b^{-1},
\eQe
as desired.
\Se

\bigskip

\Qb[Royden 1.1-3]
\Qe
\begin{solution}
Let $E$ be a nonemepty set of real numbers.\\ 
$\mathbf{( \Leftarrow )}$ Assume that 
$E$ consists of a single point, which we denote as $x$. We claim that
$\text{inf}E = x$ and $\text{sup}E = x$. As we have $x \leq x$, we see that $x$ is an upper bound
for $E$. Suppose that there exists an upper bound for $E$, $a$, that is smaller than $x$, namely 
$a < x$. This is a contradiction to the fact that $a$ is an upper bound as it is required to have
$x \leq a$ with $x \in E$. Hence, there does not exists any upper bound for $E$ that is smaller than
$x$. By definition of supremum, we have that $\text{sup}E = x$. By symmetry, we can see that
$\text{inf}E = x$ as well. Therefore, $\text{inf}E = \text{sup}E$. \\ 

\smallskip

$\mathbf{( \Rightarrow )}$ Assume that $\text{inf} E = \text{sup} E$. 
Given the assumption, let us denote the infimum and supremum for $E$ as a
single real number $a$. Then, by definition of infimum, any $x$ in $E$, we have
$a \leq x$. Furthermore, by definition of supremum, any $x$ in $E$, we have $x \leq a$.
The only real number that can satisfy the two given equality is $a$ itself. We also know
that $a$ must be in $E$ as $E$ is a nonempty set of reals. Therefore, 
we have shown that $E = \{ a \}$, and that $E$ consists of a single point.
\end{solution}

\bigskip

\begin{preposition}[1-8 Variant]
The intersection of any finite collection of open sets is open; and the union of any finite
collection of closed sets is closed.
\end{preposition}
\begin{proof} Let $\{ A_k \}_{k=1}^{n}$ be a finite collection of open sets. We want to show
that $\cap_{k=1}^{n} A_k$ is open. Consider $x \in \cap_{k=1}^{n} A_k$. Since each $A_k$ is open, 
there exist $r_k$ for $1 \leq k \leq n$ such that $B(x,r_k) \subseteq A_k$. Let $r^* = 
\underset{1 \leq k 
\leq n}{\text{min}}( A_k )$. Then, $B(x,r^*) \in A_k$ for $1 \leq k \leq n$. Hence,
$B(x,r^*) \subseteq \cap_{k=1}^{n} A_k$. Therefore, there exists $r > 0$ such that
$B(x,r)$ for all points in $\cap_{k=1}^{n} A_k$ and by definition of open sets,
$\cap_{k=1}^{n} A_k$ is open. 
We have shown that the intersection of any finite collection of open sets is open. \\

\smallskip

Let $\{ B_k \}_{k=1}^{n}$ be a finite
collection of closed sets. We want to show that $\cup_{k=1}^{n} B_k$ is closed, which is 
equivalent to $(\cup_{k=1}^{n} B_k)^{C}$ being open. By DeMorgan's identity, we have that
\eQb
(\cup_{k=1}^{n} B_k)^{C} &=& \cap_{k=1}^{n} {B_k}^{C}.
\eQe
Notice that as $B_k$ is closed, ${B_k}^{C}$ is open. Then, 
the RHS expression gives us that the set under consideration is indeed a finite
intersection of open sets, which we have shown to be open. Therefore, the union 
of any finite collection of closed sets is closed. \\

\smallskip

Note that the result does not hold true for the countable collection of open sets. The minimum can
converge $0$, breaking the above proof by having $r^*$ no longer well defined. For example,
let $A_k$ be an open interval $(-\dfrac{1}{k}, \dfrac{1}{k})$. Taking the intersection of $A_k$s 
indexed by naturals, we have 
\eQb
\cap_{k=1}^{\infty} A_k &=& \{ 0 \},
\eQe
thereby yielding an intersection that is not-open as a singleton-set is not open in reals.
Later on, we will see that the intersection of any countable of open sets is
measurable, but this result
tells us that the intersection might not necessarily open.
\end{proof}

\bigskip

\begin{theorem}[1-15 The Monotone Convergence Criterion for Real Sequences]
\end{theorem}
\begin{proof}
Let $\{ a_n \}$ be an increasing sequence. Assume that the sequence is convergent. Then, 
we have shown that the sequence is bounded previously. Assume that the sequence is
bounded. Consider a set $S = \{ a_n | n \in \mathbb{R} \}$. Since $S$ is bounded, by the 
completeness axion, there exists a supremum of $S$, which we denote as $a$.
Now, we claim that
\eQb
\underset{n \to \infty }{\text{lim}} a_n &=& a.
\eQe
Let $\epsilon > 0$. Then, as $a$ is the supremum of $S$, $a - \epsilon$ is not an upper
bound of $S$. Hence, there exists an index $n^*$ such that $a - a_{n^*} < \epsilon$. As
the sequence is increasing, for $m \geq n^*$, we have $a - a_{m} < \epsilon$. Therefore,
we have shown that for any $\epsilon > 0$, there exists an index $n$ such that for $m \geq n$,
$|a - a_m| < \epsilon$, which is precisely the definition of convergence given. 
The decreasing sequence case can also be proven in a similar vein.
Hence, we have shown that if a sequence is bounded and monotone, then it is also convergent. 
\end{proof}

\begin{theorem}[1-16 The Bolzano-Weiestrass Theorem]
\end{theorem}
\begin{proof}
\end{proof}


\section{Results: Chapter III}
\begin{definition}
All the functions considered in this chapter take values in the 
extended real numbers, that is, the set $\mathbb{R} \cup \{ \pm \infty \}$.
A property is said to hold \textbf{almost everywhere} (abbreviated a.e.) on
a measurable set $E$ provided it holds on $E \sim E_0$, where $E_0$ is a subset
of $E$ for which $m(E_0)= 0$.
\end{definition}

\bigskip

\begin{preposition}[3.1 : Definition of Measurable functions] ~\\
Let the function $f$ have a measurable domain $E$. Then, the following
statements are equivalent: \\
\begin{center}
\textit{(i)} For each real number $c$, the set $\{ x \in E | f(x) > c \}$ is measurable. \\
\textit{(ii)} For each real number $c$, the set $\{ x \in E | f(x) \geq c \}$ is measurable. \\
\textit{(iii)} For each real number $c$, the set $\{ x \in E | f(x) < c \}$ is measurable. \\
\textit{(iv)} For each real number $c$, the set $\{ x \in E | f(x) \leq c \}$ is measurable.\\
\end{center}
\end{preposition}

\bigskip

\begin{definition}
An extended real-valued function $f$ is defined on $E$ is said to be \textbf{Lebesgue measurable},
or simply \textbf{measurable}, provided its domain $E$ is measurable and it satisfies one of the
four statements of Preposition $3.1$.
\end{definition}

\bigskip

\begin{preposition}[3.2 : The Inverse Image of Open is Measurable Implies $f$ Measurable] ~\\ 
Let the function $f$ be defined on a measurable set $E$. Then $f$ is measurable if and only if
for each open set $\mathcal{O}$, the inverse image of $\mathcal{O}$ under $f$, $f^{-1}(\mathcal{O})
= \{ x \in E | f(x) \in \mathcal{O} \}$, is measurable.
\end{preposition}

\bigskip 

\begin{preposition}[3.3 : Continuity and Measurability] ~\\ 
A real-valued function that is continuous on its measurable domain is measurable.
\end{preposition}

\bigskip

\begin{preposition}[3.4] ~\\
A monotone function that is defined on an interval is measurable.
\end{preposition}

\bigskip

\begin{preposition}[3.5 : ] ~\\
Let $f$ be an extended real-valued function on $E$. \\
\textit{(i)} If $f$ is measurable on $E$ and $f = g$ a.e. on $E$, then $g$ is 
measurable on $E$. \\
\textit{(ii)} For a measurable subset $D$ of $E$, $f$ is measurable on $E$, if and only if
the restrictions of $f$ to $D$ and $E \sim D$ are measurable.
\end{preposition}

\bigskip

\begin{theorem}[3.6]
Let $f$ and $g$ be measurable functions on $E$ that are finite a.e. on $E$. \\
\textit{(Linearity)} For any $\alpha$ and $\beta$,
\eQb
\alpha f + \beta g \> \> \text{is measurable on} E.
\eQe
\textit{(Product)} 
\eQb
fg \> \> \text{is measurable on} E.
\eQe
\end{theorem}

\bigskip

\begin{remark}
Many of the properties of functions considered in elementary analysis, including continuity and 
differentiability, are preserved under the operation of composition of functions. However,
the composition of measurable functions may not be measurable.
\end{remark}

\bigskip

\begin{preposition}[3.7 : Composition and Measurability] ~\\
Let $g$ be a measurable real-valued function defined on $E$ and $f$ a continuous
real-valued function defined on all of $\mathbb{R}$. Then the composition $f \circ g$ is a 
measurable function on $E$.
\end{preposition}

\bigskip

\begin{preposition}
For a finite family $\{ f_k \}_{k=1}^{n}$ of measurable functions with common domain $E$,
the functions max$\{ f_1 , ..., f_n \}$ and min$\{ f_1 , ..., f_n \}$ also are measurable.
\end{preposition}

\bigskip

\begin{remark}
For a function $f$ defined on $E$, we have the associated functions $|f|, f^{+}, f^{-}$ defined
on $E$ by 
\eQb
|f|(x) = max \{ f(x), -f(x) \} , f^{+}(x) = max \{ f(x), 0 \}, f^{-}(x) = max \{ -f(x), 0 \} .
\eQe
If $f$ is measurable on $E$, then, by the preceding proposition, so are the functions $|f|$, 
$f^{+}$, and $f^{-}$. This will be important when we study integration since the expression of $f$
as the difference of two nonnegative functions,
\eQb
f &=& f^{+} - f^{-} \> \text{on } E,
\eQe
plays an important part in defining the Lebesgue integral.
\end{remark}

\bigskip

\begin{preposition}
Let $\{ f_n \} $ be a sequence of measurable functions on $E$ that converges pointwise 
a.e. on $E$ to the function $f$. Then, $f$ is measurable.
\end{preposition}

\bigskip

\begin{definition}
If $A$ is any set, the \textbf{characteristic function} of $A$, $\mathcal{X}_{A}$,
is the function on $\mathbb{R}$ defined by 
\eQb
\mathcal{X}_{A} &=& 1,
\eQe
It is clear that the function $\mathcal{X}_{A}$ is measurable if and only if the set $A$
is measurable. Thus, the existence of a non-measurable set implies the existence of
a non-measurable function. Linear combinations of characteristic functions of measurable sets
play a role in Lebesgue integration similar to that played by step functions in Riemann
integration, and so we name these functions.
\end{definition}

\bigskip

\begin{definition}
A real-valued function $\varphi$ defined on a measurable set $E$ is called \textbf{simple} 
provided it is measurable and takes only a finite number of values. \\
\\
We emphasize that a simple function only takes real values. Linear combinations and
products of simple functions are simple since each of them takes on only a finite
number of values. If $\varphi$ is simple, has domain $E$ and takes the distinct values
$c_1, .., c_n$, then 
\eQb
\varphi &=& \sum_{k=1}^{n} c_k \cdot \mathcal{X}_{E_k} \> \text{on} \> E, 
\> \text{where} \> E_k = \{ x \in E | \varphi(x) = c_k \} .
\eQe
This particular expression of $\varphi$ as a linear combination of characteristic functions
is called the \textbf{canonical representation of the simple function} $\varphi$.
\end{definition}

\bigskip

\begin{lemma}[ : The Simple Approximation Lemma]
Let $f$ be a measurable real-valued function on $E$. Assume $f$ is bounded on $E$,
that is, there is an $M \geq 0$ for which $|f| \leq M$ on $E$. Then, for each

\end{lemma}


\end{document}
