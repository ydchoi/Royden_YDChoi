\documentclass{article} % For LaTeX2e
\usepackage{nips14submit_e,times}
\usepackage{amsmath}
\usepackage{amsthm}
\usepackage{amssymb}
\usepackage{mathtools}
\usepackage{hyperref}
\usepackage{url}
\usepackage{algorithm}
\usepackage[noend]{algpseudocode}
%\documentstyle[nips14submit_09,times,art10]{article} % For LaTeX 2.09

\usepackage{graphicx}
\usepackage{caption}
\usepackage{subcaption}

\def\eQb#1\eQe{\begin{eqnarray*}#1\end{eqnarray*}}
\def\eQnb#1\eQne{\begin{eqnarray}#1\end{eqnarray}}
\providecommand{\e}[1]{\ensuremath{\times 10^{#1}}}
\providecommand{\pb}[0]{\pagebreak}


\def\Qb#1\Qe{\begin{question}#1\end{question}}
\def\Sb#1\Se{\begin{solution}#1\end{solution}}

\newenvironment{claim}[1]{\par\noindent\underline{Claim:}\space#1}{}
\newtheoremstyle{quest}{\topsep}{\topsep}{}{}{\bfseries}{}{ }{\thmname{#1}\thmnote{ #3}.}
\theoremstyle{quest}
\newtheorem*{definition}{Definition}
\newtheorem*{theorem}{Theorem}
\newtheorem*{question}{Question}
\newtheorem*{preposition}{Preposition}
\newtheorem*{exercise}{Exercise}
\newtheorem*{challengeproblem}{Challenge Problem}
\newtheorem*{solution}{Solution}
\newtheorem*{remark}{Remark}
\usepackage{verbatimbox}
\usepackage{listings}
\title{Royden}


\author{
Youngduck Choi \\
Courant Institute of Mathematical Sciences \\
New York University \\
\texttt{yc1104@nyu.edu} \\
}


% The \author macro works with any number of authors. There are two commands
% used to separate the names and addresses of multiple authors: \And and \AND.
%
% Using \And between authors leaves it to \LaTeX{} to determine where to break
% the lines. Using \AND forces a linebreak at that point. So, if \LaTeX{}
% puts 3 of 4 authors names on the first line, and the last on the second
% line, try using \AND instead of \And before the third author name.

\newcommand{\fix}{\marginpar{FIX}}
\newcommand{\new}{\marginpar{NEW}}

\nipsfinalcopy % Uncomment for camera-ready version

\begin{document}


\maketitle

\begin{abstract}
This work contains the solutions to Royden's Real Variables.
\end{abstract}

\section{Chapter II}
\Qb[Royden 2.1-1]
\Qe
\begin{solution}
Let $m$ be a set function defined for all sets in a $\sigma-$algebra $\mathcal{A}$ 
with values in $[0, \infty]$. Assume that $m$ is countably additive over countable 
disjoint collections of sets in $\mathcal{A}$. Furthermore, 
assume that $A$ and $B$ are two sets in
$\mathcal{A}$ with $A \subseteq B$. 
Given that $m$ is countably additive over countable disjoint collections of sets in
$\mathcal{A}$, we have 
\eQb
m(B) &=& m(A) + m(B \setminus A),
\eQe
where $B \setminus A$ is a well-defined set with $A \subseteq B$ assumption, thus
$A$ and $B \setminus A$ forming a valid countable disjoint collections of sets 
whose union is $B$.
With $m$ being
a set function with values in $[0, \infty]$, we obtain
$m(B) = m(A) + r$,
where $r$ denotes some non-negative real value. Therefore, we finally get
\eQb
m(A) \leq m(B).
\eQe
Hence, we have shown that the given set function $m$ has the monotonicity 
property.
\end{solution}

\bigskip

\Qb[Royden 2.1-2]
\Qe
\begin{solution}

Let $m$ be a set function defined for all sets in a $\sigma-$algebra $\mathcal{A}$ 
with values in $[0, \infty]$. Assume that $m$ is countably additive over countable 
disjoint collections of sets in $\mathcal{A}$. Furthermore, assume that there exists 
a set $A$ in the collection $\mathcal{A}$ such that $m(A) < \infty$. Using the countably
additive property with a collection $\{ A, \emptyset \}$, we obtain
\eQb
m(A \cup \emptyset ) &=& m(A) + m(\emptyset).
\eQe
Substituting $A \cup \emptyset = A$ and subtracting $m(A)$ from both sides, granted with finiteness of
$m(A)$, we get
\eQb
m(\emptyset ) &=& 0,
\eQe
as desired. Hence, we have shown that if there is a set $A$ in the collection $\mathcal{A}$ for
whcih $m(A) < \infty$, then $m( \emptyset ) = 0$.
\end{solution}

\bigskip

\Qb[Royden 2.1-3]
\Qe
\begin{solution}

\end{solution}

\bigskip

\Qb[Royden 2.1-5 (A Countable Set Has Outer Measure Zero)]
\Qe
\begin{solution}
We know that any countable set has outer measure zero.
Using the fact that the outer measure of an interval is its length yields $m^{*}( [0,1] ) = 1$.
Therefore, $[0,1]$ cannot countable.
\end{solution}

\bigskip

\Qb[: Royden 2.1-6]
\Qe
\begin{solution}
Let $Q$ and $A$ denote the set of rationals and irrationals in the interval $[0,1]$
respectively. Consider a countable collection of sets $\{ Q, A \}$.
Since outer measure is countably subadditive, we have 
\eQb
m^{*}( Q \cup A) & \leq & m^*(Q) + m^{*}(A).
\eQe
As $Q$ is a countable set whose outer measure is zero 
and $Q \cup A = [0,1]$ by construction, we obtain
\eQb
m^{*}([0,1]) & \leq & m^{*}(A).
\eQe
As the outer measure of an interval is its length, we have
\eQb
1 & \leq & m^{*}(A).
\eQe
Using the monotonicity property of outer measure with $I \subset [0,1]$,
we also see
\eQb
m^{*}(A) & \leq & 1,
\eQe
thereby showing that $m^{*}(A) = 1$. 
\end{solution}

\bigskip

\Qb[: Royden 2.3-Proposition 4] 
\Qe
\begin{solution}
We want to show that any set of outer measure zero is measurable, which further implies that
any countable set is measurable. Let the set $E$ to have outer measure zero, $m^{*}(E) = 0$.
Let $A$ be any set. Since
\eQb
A \cap E \subseteq E \> \text{ and } \> A \cap E^c \subseteq A,
\eQe
by the monotonicity of outer measure, we obtain
\eQb
m^{*}(A \cap E) \leq m^{*}(E) = 0 \> \text{ and } \> m^{*}(A \cap E^c) \leq m^{*}(A).
\eQe
It is important to note that it suffices to show the above statement to show that the set $E$
is measurable, as the inequality,
\eQb
m^{*} \leq m^{*}(A \cap E) + m^{*}( A \cap E^c),
\eQe
trivially holds with the finite subadditive property of outer measure.
\end{solution}

\bigskip

\begin{preposition}
Algebra is closed with respect to the formation of finite intersection.
\end{preposition}
\begin{proof}
Let $\mathcal{A}$ be an algebra and $\{ A_k \}_{k=1}^{n}$ be a finite collection of
sets in $\mathcal{A}$. We want to show that $\cap_{k=1}^{n} A_k$ is in $\mathcal{A}$.
Since algebra is closed with respect to the formation of finite union
and complements, we have 
\eQb
\cup_{k=1}^{n} A_k^{C} \> \text{ is in} \> \mathcal{A}. 
\eQe
By applying De Morgan's identity and closedness of complements iteratively, we see that
\eQb
{(\cap_{k=1}^{n} A_k)}^{C} \> \text{ is in} \> \mathcal{A} \>\> \text{ and } \>\>
\cap_{k=1}^{n} A_k \> \text{ is in} \> \mathcal{A}, 
\eQe
thereby showing that an algebra is closed with respect to the formation of finite intersection.
\end{proof}

\section{Chapter I}

\Qb[: Royden 1.1-1 (Distributive Property of Multiplicative Inverse in Reals)]
\Qe
\Sb
Assume that $a \neq 0$ and $b \neq 0$. From the multiplicative identity axiom,
we have that a multiplicative inverse exists for $a$ and $b$ individually, which we denote as
$a^{-1}$ and $b^{-1}$ respectively.
Now, consider the expression $(ab)(a^{-1}b^{-1})$, where $ab$ denotes the 
product of $a$ and $b$, and $a^{-1}b^{-1}$ denotes the product of $a^{-1}$ and $b^{-1}$.
From the commutativity of multiplication,
we obtain
\eQb
(ab)(a^{-1}b^{-1}) = (ab)(b^{-1}a^{-1}).
\eQe
Using the associativity of multiplication and iteratively substituting $bb^{-1} = 1$ and $aa^{-1} = 1$,
we have
\eQb
(ab)(a^{-1}b^{-1}) = 1,
\eQe
where $1$ denotes the identity as usual. Hence, the product, $a^{-1}b^{-1}$ satisfies definition
of multiplicative inverse with respect to the $ab$ term whose multiplicative inverse can be denoted
as $(ab)^{-1}$ by convention. Therefore, we obtain that
\eQb
(ab)^{-1} = a^{-1}b^{-1},
\eQe
as desired.
\Se

\bigskip

\Qb[Royden 1.1-3]
\Qe
\begin{solution}
Let $E$ be a nonemepty set of real numbers.\\ 
$\mathbf{( \Leftarrow )}$ Assume that 
$E$ consists of a single point, which we denote as $x$. We claim that
$\text{inf}E = x$ and $\text{sup}E = x$. As we have $x \leq x$, we see that $x$ is an upper bound
for $E$. Suppose that there exists an upper bound for $E$, $a$, that is smaller than $x$, namely 
$a < x$. This is a contradiction to the fact that $a$ is an upper bound as it is required to have
$x \leq a$ with $x \in E$. Hence, there does not exists any upper bound for $E$ that is smaller than
$x$. By definition of supremum, we have that $\text{sup}E = x$. By symmtry, we can see that
$\text{inf}E = x$ as well. Therefore, $\text{inf}E = \text{sup}E$. \\ 

\smallskip

$\mathbf{( \Rightarrow )}$ Assume that $\text{inf} E = \text{sup} E$. 
Given the assumption, let us denote the infimum and supremum for $E$ as a
single real number $a$. Then, by definition of infimum, any $x$ in $E$, we have
$a \leq x$. Furthermore, by definition of supremum, any $x$ in $E$, we have $x \leq a$.
The only real number that can satisfy the two given equality is $a$ itself. We also know
that $a$ must be in $E$ as $E$ is a nonempty set of reals. Therefore, 
we have shown that $E = \{ a \}$, and that $E$ consists of a single point.
\end{solution}




\end{document}
