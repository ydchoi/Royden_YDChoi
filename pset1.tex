\documentclass{article} % For LaTeX2e
\usepackage{nips14submit_e,times}
\usepackage{amsmath}
\usepackage{amsthm}
\usepackage{amssymb}
\usepackage{mathtools}
\usepackage{hyperref}
\usepackage{url}
\usepackage{algorithm}
\usepackage[noend]{algpseudocode}
%\documentstyle[nips14submit_09,times,art10]{article} % For LaTeX 2.09

\usepackage{graphicx}
\usepackage{caption}
\usepackage{subcaption}

\def\eQb#1\eQe{\begin{eqnarray*}#1\end{eqnarray*}}
\def\eQnb#1\eQne{\begin{eqnarray}#1\end{eqnarray}}
\providecommand{\e}[1]{\ensuremath{\times 10^{#1}}}
\providecommand{\pb}[0]{\pagebreak}


\def\Qb#1\Qe{\begin{question}#1\end{question}}
\def\Sb#1\Se{\begin{solution}#1\end{solution}}

\newenvironment{claim}[1]{\par\noindent\underline{Claim:}\space#1}{}
\newtheoremstyle{quest}{\topsep}{\topsep}{}{}{\bfseries}{}{ }{\thmname{#1}\thmnote{ #3}.}
\theoremstyle{quest}
\newtheorem*{definition}{Definition}
\newtheorem*{theorem}{Theorem}
\newtheorem*{lemma}{Lemma}
\newtheorem*{question}{Question}
\newtheorem*{preposition}{Preposition}
\newtheorem*{exercise}{Exercise}
\newtheorem*{challengeproblem}{Challenge Problem}
\newtheorem*{solution}{Solution}
\newtheorem*{remark}{Remark}
\usepackage{verbatimbox}
\usepackage{listings}

\title{Real Variables: \\
Problem Set I}


\author{
Youngduck Choi \\
Courant Institute of Mathematical Sciences \\
New York University \\
\texttt{yc1104@nyu.edu} \\
}


% The \author macro works with any number of authors. There are two commands
% used to separate the names and addresses of multiple authors: \And and \AND.
%
% Using \And between authors leaves it to \LaTeX{} to determine where to break
% the lines. Using \AND forces a linebreak at that point. So, if \LaTeX{}
% puts 3 of 4 authors names on the first line, and the last on the second
% line, try using \AND instead of \And before the third author name.

\newcommand{\fix}{\marginpar{FIX}}
\newcommand{\new}{\marginpar{NEW}}

\nipsfinalcopy % Uncomment for camera-ready version

\begin{document}


\maketitle

\begin{abstract}
This work contains the solutions to the first problem set of Real Variables 2015.
\end{abstract}

\section{Solutions}
\begin{question}[1. Royden 2.4. Counting Measure]
\end{question}
\begin{solution}
We wish to show that the counting measure, $c: \mathcal{P}(\mathbb{R}) 
\to [0,\infty]$, where $\mathcal{P}(\mathbb{R})$ denotes the power set of $\mathbb{R}$, is
countably additive and translation invariant. \\ 

We first prove that it is countably additive. Let $\{ E_k \}_{k=1}^{\infty}$ be a countable,
disjoint collection of subsets of $\mathbb{R}$. 
If one of the set in the collection has infinite cardinality, then 
we have  
\begin{eqnarray*}
\sum_{k=1}^{\infty} c(E_k) &=& \infty,
\end{eqnarray*}
as $c(E_k) = \infty$ for some $k$. Notice that the union of the collection $\cup_{k=1}^{\infty} E_k$,
also has infinite cardinality, as it has a subset with an infinite cardinality. Hence, by the definition
of counting measure, we have
$c(\cup_{k=1}^{\infty}E_k) = \infty$. Therefore, we have
\begin{eqnarray*}
c(\cup_{k=1}^{\infty}E_k) = \sum_{k=1}^{\infty} c(E_k),
\end{eqnarray*}
for the case under consideration. Now, assume that $c(E_k) < \infty$ for all $k$. There are
two sub-cases now. First, assume that the series $\sum_{k=1}^{\infty} c(E_k )$ converges. 
In particular, we have that $\underset{k \to \infty}{\lim} c(E_k ) = 0$. For some 
$\epsilon < 1$,
we have an $N$ such that $c(E_k ) < \epsilon$ for all $k \geq N$. As the counting measure only
takes an integer value or $\infty$, we obtain that $c(E_k) = 0$ and $E_k = \emptyset$
for all $k \geq N$. Furthermore, we get that  
\begin{eqnarray*}
c(\cup_{k=1}^{\infty} E_k) &=& c(\cup_{k=1}^{N}E_k), \\ 
\sum_{k=1}^{\infty} c(E_k) &=& \sum_{k=1}^{N} c(E_k).
\end{eqnarray*}

\pagebreak

As the finite additivity of counting measure trivially holds (can be shown
with a simple inductive argument),
$c(\cup_{k=1}^{N} E_k) = \sum_{k=1}^{N} c(E_k)$ holds, and thus we conclude that
\eQb
c(\cup_{k=1}^{\infty} E_k) &=& \sum_{k=1}^{\infty} c(E_k).
\eQe
Now, for the last case, assume that $\sum_{k=1}^{\infty} c(E_k ) = \infty$. This gives that
for any $N$, there exists $N' > N$, such that $c(E_{N'}) > 0$. Hence, there exists a subsequence
of $\{ E_k \}_{k \in \lambda}$, 
which only consists of disjoint nonempty subsets. 
Therefore, $\cup_{k=1}^{\infty} E_k$ contains a countably infinite
sub-collection of nonempty disjoint sets. Hence,  
the set $\cup_{k=1}^{\infty} E_k$ must contain infinitely many members. Therefore, we have
$c(\cup_{k=1}^{\infty} E_k) = \infty$, thus,
\[
c(\cup_{k=1}^{\infty} E_k) = \sum_{k=1}^{\infty} c(E_k).
\]
As we covered all possible cases, this completes the proof for the countable additivity of 
counting measure. $\qed$
\\
\smallskip

We now prove the translation invariant property of counting measure. Consider a set $E$ and its 
translation defined by $E+y = \{ x + y | x \in E \}$. Notice that 
a map, $\phi:\mathbb{R} \to \mathbb{R}$ where $\phi(x) = x+y$, establishes a bijection
between $E$ and $E+y$. Hence, the cardinality
of $E$ and $E+y$ are equal. As the equal cardinality is a sufficient condition(in fact, strictly more) 
for the equal counting measure, we have shown that the counting measure
is translation invariant. $\qed$


\end{solution}


\bigskip

\begin{question}[2. Royden 2.8]
\end{question}
\begin{solution}
Before we proceeding to the main part of the proof, we first prove the following lemma.
\begin{lemma}
Let $\{ X_k \}_{k=1}^{n}$ be a finite collection of sets of real numbers. Then, we have that
\[
\overline{\cup_{k=1}^{n} X_k } = \cup_{k=1}^{n} \overline{X_k},
\]
where $\overline{X_k}$ denotes a closure of the set $X_k$.
\end{lemma}
\begin{proof}
Let $X_1$ and $X_2$ be sets of real numbers. Then,
we have $X_1 \subseteq X_1 \cup X_2$ and $X_2 \subseteq X_1 \cup X_2$. Consequently, we have
$\overline{X_1} \subseteq \overline{X_1 \cup X_2}$ and $\overline{X_2} \subseteq \overline{X_1 \cup X_2}$.
Hence, we have $\overline{X_1} \cup \overline{X_2} = \overline{X_1 \cup X_2}$. 
This can be trivially extended
to any finite $n$ with an induction using set union, i.e. take $\cup_{k=1}^{n-1} X_k$ with $X_n$.
Therefore, $\cup_{k=1}^{n} \overline{X_k} \subseteq \overline{\cup_{k=1}^{n} X_k}$. \\

Now, notice that $\overline{X_1} \cup \overline{X_2}$ is closed, as it is a finite union of closed sets.
Since $X_1 \subseteq \overline{X_1}$ and $X_2 \subseteq \overline{X_2}$, we have $X_1
\cup X_2 \subseteq \overline{X_1} \cup \overline{X_2}$. As $\overline{X_1} \cup \overline{X_2}$ is
closed, we obtain $\overline{X_1 \cup X_2} \subseteq \overline{X_1} \cup \overline{X_2}$. Again,
this argument can be extended inductively for a finite $n$, by taking $\cup_{k=1}^{n-1} X_k$ and
$X_n$. Therefore, $\overline{\cup_{k=1}^{n} X_k} \subseteq \cup_{k=1}^{n} \overline{X_k}$.
This concludes
that $\cup_{k=1}^{n} \overline{X_k} = \overline{\cup_{k=1}^{n} X_k}$ for $n$ finite.
One should note that this argument fails with $n$ infinite, as we no longer can leverage the fact
that finite union of closed sets is closed.
\end{proof}


We now proceed to the main part of the proof. 
Let $B$ be a set of rational numbers in the interval $[0,1]$, and let $\{ I_k \}_{k=1}^{n}$ be a 
finite collection of open intervals that cover $B$.
As $B \subseteq \cup_{k=1}^{n} I_k$, we have $\overline{B} \subseteq 
\overline{\cup_{k=1}^{n}I_k}$. Furthermore, with $n$ 
being finite, we obtain that $\overline{\cup_{k=1}^{n} I_k} =
\cup_{k=1}^{n} \overline{I_k}$. Then, it follows from the monotonicity, and finite sub-additivity
property that 
\begin{equation}
m^{*}(\overline{B}) \leq m^{*}( \cup_{k=1}^{n} \overline{ I_k } ) \leq \sum_{k=1}^{n} 
m^{*}(\overline{I_k}).
\end{equation}
In particular, we have $m^{*}(\overline{B}) = 1$, as $B = [0,1]$, and 
$\sum_{i=1}^{n} m^{*}(\overline{I_k}) = \sum_{i=1}^{n} m^{*}(I_k)$, as the outer measure
of an open interval and corresponding closed interval are equal. Substituting the two
equalities into the (1) inequality, we obtain
\eQb
\sum_{i=1}^{n} m^{*}(I_k) \geq 1,
\eQe
as desired. $\qed$
\end{solution}

\pagebreak

\begin{question}[3. Royden 2.14]
\end{question}
\begin{solution}
Let $m^{*}(E) > 0$. We wish to find a subset $X$ of $E$ such that $m^{*}(X) > 0$. 
Consider the countable collection of sets $\{ (-M,M) \}_{M=1}^{\infty}$. Notice that,
as $(-M,M)$ is bounded for some fixed $M$,
$E \cap (-M,M)$ is a bounded subset of $E$. Furthermore,
$ E = \cup_{M=1}^{\infty} E\cap (-M,M)$.
Then, by the countable sub-additivity of outer measure, we have
\eQb
\sum_{M=1}^{\infty} m^{*}(E \cap (-M,M)) \geq m^{*}(E).
\eQe
If $m^{*}(E \cap (-M,M)) = 0$ for all $M$, then we have the sum on the LHS equals $0$, and obtain
$0 > 0$, as $m^{*}(E) > 0$. This is a contradiction. Hence, there exists a $M$ such that
$m^{*}(E \cap (-M,M)) > 0$, and $E \cap (-M,M)$ is precisely the bounded subset of $E$ with
positive outer measure. We have shown that if a set $E$ has positive outer measure,
then there is a bounded subset of $E$ that also has positive outer measure. $\qed$

\end{solution}

\bigskip

\begin{question}[4. Royden 2.15]
\end{question}
\begin{solution}
Let $m(E) < \infty$ and $\epsilon > 0$. We wish to show that $E$ is the disjoint
union of a finite number of measurable sets, each of which has measure at most $\epsilon$.
First, assume that $E$ is bounded.
Then, there exists an interval $[-M,M]$ such that
$E \subseteq [-M,M]$. By the Archimedean principle, there exists $N \in \mathbb{N}$ such that
$\dfrac{2M}{N} < \epsilon$. Now, consider the following finite disjoint collection of sets:
\eQb
\{ [-M+\dfrac{2M}{N}(n-1),-M + \dfrac{2M}{N}n) \cap E \}_{n=1}^{N+1}.
\eQe
Notice that $E = \cup_{n=1}^{N+1}
[-M+\dfrac{2M}{N}(n-1), -M+\dfrac{2M}{N}n) \cap E$. Furthermore, as every interval is measurable
and intersection of two measurable sets is measurable,
each set in the collection is measurable. By the monotonicity property of measure,
we have 
\eQb
m([-M+\dfrac{2M}{N}(n-1),-M+\dfrac{2M}{N}n) \cap E) &\leq & 
m([-M+\dfrac{2M}{N}(n-1),-M+\dfrac{2M}{N}n)) \\
&=& \dfrac{2M}{N} \\ 
&<& \epsilon.
\eQe
Hence,
we have found a finite disjoint collection of measurable sets, each of which has measure at
most $\epsilon$, whose union equals $E$.
We have proven the statement for the $E$ bounded case.

\smallskip

Now assume that $E$ is unbounded. 
Notice that $E =\cup_{M=1}^{\infty} E \cap [-M,M]$. As $\{ E \cap [-M,M] \}_{M=1}^{\infty}$
forms an ascending
collection of measurable sets, by the continuity of measure, we have
\eQb
m(E) &=& \underset{M \to \infty}{\lim} m(E \cap [-M,M]).
\eQe
As $m(E) < \infty$, there exists an $N$ such that
\eQnb
m(E) - \epsilon < m(E \cap [-N,N]) \leq m(E).
\eQne
Note that, by the finite additivity of measure, we have
\eQnb
m(E \cap [-N,N]) + m(E \setminus [-N,N] = m(E). 
\eQne
Combining $(2)$ and $(3)$, we obtain that $m(E \setminus [-N,N]) \leq \epsilon$.
Now, as $E \cap [-N,N]$ is bounded,
there exists a finite collection of measurable sets $\{E_i \}_{i=1}^{K}$
such that $m(E_i) \leq \epsilon$ for $1 \leq i \leq k$, and $\cup_{i=1}^{K} E_i = E \cap [-N,N]$.
Notice that $E = \cup_{i=1}^{K} E_i \cup (E \setminus [-N,N])$. In particular, the 
collection
$\{ \{E_i\}_{i=1}^{K}, E \setminus [-N,N] \}$ is a finite disjoint 
collection of measurable sets, whose union equals $E$, 
each of which has measure at most $\epsilon$. This proves the statement
for the $E$ unbounded case, and we have completed the proof. $\qed$
\end{solution}

\pagebreak

\begin{question}[5. Royden 2.17]
\end{question}
\begin{solution}
From the inner approximation by closed sets,
and outer approximation by open sets, 
we have that $E$ is measurable if and only if, for any $\epsilon > 0$,
there exists a closed set $F$ and an open set $O$, such that
\eQb
E \subseteq O \>\> \text{with } \>\> m^{*} (O \setminus E) < \dfrac{\epsilon}{2} 
\>\> \text{and} \>\>
F \subseteq E \>\>  \text{with} \>\> m^{*} (E \setminus F) < \dfrac{\epsilon}{2}. 
\eQe
Applying the sub-additivity property of outer measure with $O\setminus E$ and $E \setminus F$,
we have
\eQb
m^{*}( O \setminus F) < m^{*}(O \setminus E) + m^{*}(E \setminus F) < \epsilon.
\eQe
Hence, $E$ is measurable, if and only if, there exists an open set $O$ and a closed set $F$ for which
$F \subseteq E \subseteq O$ and $m^{*}(O \setminus F) < \epsilon$. $\qed$
\end{solution}

\bigskip

\begin{question}[6. Royden 2.28]
\end{question}
\begin{solution}
Let $\{ E_k \}_{k=1}^{\infty}$ be a countable, disjoint collection of measurable sets.
By the finite additivity property, we have 
\[
m( \cup_{k=1}^{N} E_k ) = \sum_{k=1}^{N} m(E_k),
\]
for all $N$. Notice that 
$\{ \cup_{k=1}^{N} E_k \}_{N=1}^{\infty}$ forms
an ascending collection of measurable sets. 
Hence, by applying the continuity of measure to the ascending collection,
$\{ \cup_{k=1}^{N} E_k \}_{N=1}^{\infty}$, we have
\eQb
m( \cup_{N=1}^{\infty} \cup_{k=1}^{N} E_k ) &=& \underset{N \to \infty}{\lim} 
m(\cup_{k=1}^{N}E_k).
\eQe
Simplifying the LHS and applying the finite additivity property to the RHS, we obtain
\eQb
m(\cup_{k=1}^{\infty} E_k) = \sum_{k=1}^{\infty} m(E_k).
\eQe
Since $\{E_k \}_{k=1}^{\infty}$
was chosen to be an arbitrary countable, disjoint collection of measurable sets, we have
shown that finite additivty and continuity of measure implies countable additivity. $\qed$ 

\bigskip

\end{solution}




















\end{document}








