\documentclass{article} % For LaTeX2e
\usepackage{nips14submit_e,times}
\usepackage{amsmath}
\usepackage{amsthm}
\usepackage{amssymb}
\usepackage{mathtools}
\usepackage{hyperref}
\usepackage{url}
\usepackage{algorithm}
\usepackage[noend]{algpseudocode}
%\documentstyle[nips14submit_09,times,art10]{article} % For LaTeX 2.09

\usepackage{mathrsfs}
\usepackage{graphicx}
\usepackage{caption}
\usepackage{subcaption}

\def\eQb#1\eQe{\begin{eqnarray*}#1\end{eqnarray*}}
\def\eQnb#1\eQne{\begin{eqnarray}#1\end{eqnarray}}
\providecommand{\e}[1]{\ensuremath{\times 10^{#1}}}
\providecommand{\pb}[0]{\pagebreak}


\def\Qb#1\Qe{\begin{question}#1\end{question}}
\def\Sb#1\Se{\begin{solution}#1\end{solution}}

\newenvironment{claim}[1]{\par\noindent\underline{Claim:}\space#1}{}
\newtheoremstyle{quest}{\topsep}{\topsep}{}{}{\bfseries}{}{ }{\thmname{#1}\thmnote{ #3}.}
\theoremstyle{quest}
\newtheorem*{definition}{Definition}
\newtheorem*{theorem}{Theorem}
\newtheorem*{lemma}{Lemma}
\newtheorem*{question}{Question}
\newtheorem*{preposition}{Preposition}
\newtheorem*{exercise}{Exercise}
\newtheorem*{challengeproblem}{Challenge Problem}
\newtheorem*{solution}{Solution}
\newtheorem*{remark}{Remark}
\usepackage{verbatimbox}
\usepackage{listings}

\title{Real Variables: \\
Problem Set II}


\author{
Youngduck Choi \\
Courant Institute of Mathematical Sciences \\
New York University \\
\texttt{yc1104@nyu.edu} \\
}


% The \author macro works with any number of authors. There are two commands
% used to separate the names and addresses of multiple authors: \And and \AND.
%
% Using \And between authors leaves it to \LaTeX{} to determine where to break
% the lines. Using \AND forces a linebreak at that point. So, if \LaTeX{}
% puts 3 of 4 authors names on the first line, and the last on the second
% line, try using \AND instead of \And before the third author name.

\newcommand{\fix}{\marginpar{FIX}}
\newcommand{\new}{\marginpar{NEW}}

\nipsfinalcopy % Uncomment for camera-ready version

\begin{document}


\maketitle

\begin{abstract}
This work contains solutions to the problem set III of Real Variables 2015 at NYU.
\end{abstract}

\section{Solutions}

\bigskip

\begin{question}[1. Royden 3.20]
\end{question}
\begin{solution}
\eQb
\chi_{A \cap B} &=& \begin{cases} 1 &\mbox{if } x \in A \cap B \\ 
0 & \mbox{if } x \notin A \cap B \end{cases}\\ 
\chi_{A \cup B} &=& \begin{cases} 1 &\mbox{if } x \in A \cup B \\ 
0 & \mbox{if } x \notin A \cup B \end{cases} \\
\chi_{A^c} &=& \begin{cases} 1 &\mbox{if } x \in A^c \\ 
0 & \mbox{if } x \notin A^c \end{cases} \\
\eQe
\end{solution}

\bigskip

\begin{question}[2. Royden 3.21]
\end{question}
\begin{solution}
\end{solution}

\bigskip

\begin{question}[3. Royden 3.27]
\end{question}
\begin{solution}
The Egoroff 
\end{solution}

\bigskip

\begin{question}[4. Royden 4.12]
\end{question}
\begin{solution}
Let $f$ a bounded measurable function on a set of finite measure $E$. Assume $g$ is bounded
and $f = g$ a.e. on $E$. First, we note that $g$ is measurable. Since both $f$ and $g$
are bounded measurable functions, we have $\int_{E} f$ and $\int_{E} g$ well-defined.
Let $\epsilon > 0$, and $E_0 = \{ x \in E \> | \> f(x) \neq g(x) \}$. Note that 
$\mathrm{m}(E_0) = 0$, as $f=g$ a.e. Consequently, $E \setminus E_0$ and $E_0$ are 
disjoint measurable sets.  Then, by additivity over domain and linearity of integration,
we have 
\eQb
\left| \int_{E}f - \int_{E} g \right| &=& 
\left| \int_{E \setminus E_0}f - \int_{E \setminus E_0}g 
 + \int_{E_0}f - \int_{E_0}g \right| \\
&=& \left| \int_{E \setminus E_0} f - g + 
\int_{E_0} f - g \right|.
\eQe
As $f = g$ on $E \setminus E_0$, and $\int_{E \setminus E_0} f - g = 0$, and consequently,
\eQb
\left| \int_{E}f - \int_{E} g \right| &=&  
\left| \int_{E_0} f - g \right|.
\eQe
ddd,
\eQb
\int_{E}f = \int_{E}g,
\eQe 
as desired. $\qed$
\end{solution}

\bigskip

\begin{question}[5. Royden 4.23]
\end{question}
\begin{solution}
Let $\{ a_n \}$ be a sequence of nonnegative real numbers. Define the function $f$ on 
$E = [1,\infty )$ setting $f(x) = a_n$ if $ n \leq x < n+1$. We first show that 
$f$ is measurable. Let $c \in \mathbb{R}$. Then, we can express the pre-image set of $c$ as
\eQb
\{ x \in E \> | \> f(x) = c \} &=& \bigcup_{k \in \lambda}[k,k+1),
\eQe
where $\lambda = \{ k \in \mathbb{N} \> | \> a_k = c \}$. Therefore, $\{ x \in E \> | \> 
f(x) = c \}$ is measurable, as it is either an empty set or 
a countable union of collection of intervals of the form $[k,k+1)$, which are measurable. Since 
$c$ was arbitrary, $f$ is measurable.
Then, by the additivity over domain of integration property, we have
\eQb
\int_{E} f &=& \sum_{k=1}^{\infty} \int_{I_k} f,
\eQe
where $I_k$ denotes $[k,k+1)$. As $f(I_k) = a_k$ and by the linearity of integration, 
we obtain
\eQb
\int_{E} f &=& \sum_{k=1}^{\infty} \int_{I_k} a_k \\
&=& \sum_{k=1}^{\infty} a_k \int_{I_k} 1 \\
&=& \sum_{k=1}^{\infty} a_k.
\eQe
Therefore, we have shown that $\int_{E} f = \sum_{k=1}^{\infty} a_k$. $\qed$
\end{solution}

\bigskip

\begin{question}[6. Royden ]
\end{question}
\begin{solution}
\end{solution}

\end{document}
