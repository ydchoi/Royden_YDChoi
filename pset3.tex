\documentclass{article} % For LaTeX2e
\usepackage{nips14submit_e,times}
\usepackage{amsmath}
\usepackage{amsthm}
\usepackage{amssymb}
\usepackage{mathtools}
\usepackage{hyperref}
\usepackage{url}
\usepackage{algorithm}
\usepackage[noend]{algpseudocode}
%\documentstyle[nips14submit_09,times,art10]{article} % For LaTeX 2.09

\usepackage{mathrsfs}
\usepackage{graphicx}
\usepackage{caption}
\usepackage{subcaption}

\def\eQb#1\eQe{\begin{eqnarray*}#1\end{eqnarray*}}
\def\eQnb#1\eQne{\begin{eqnarray}#1\end{eqnarray}}
\providecommand{\e}[1]{\ensuremath{\times 10^{#1}}}
\providecommand{\pb}[0]{\pagebreak}


\def\Qb#1\Qe{\begin{question}#1\end{question}}
\def\Sb#1\Se{\begin{solution}#1\end{solution}}

\newenvironment{claim}[1]{\par\noindent\underline{Claim:}\space#1}{}
\newtheoremstyle{quest}{\topsep}{\topsep}{}{}{\bfseries}{}{ }{\thmname{#1}\thmnote{ #3}.}
\theoremstyle{quest}
\newtheorem*{definition}{Definition}
\newtheorem*{theorem}{Theorem}
\newtheorem*{lemma}{Lemma}
\newtheorem*{question}{Question}
\newtheorem*{preposition}{Preposition}
\newtheorem*{exercise}{Exercise}
\newtheorem*{challengeproblem}{Challenge Problem}
\newtheorem*{solution}{Solution}
\newtheorem*{remark}{Remark}
\usepackage{verbatimbox}
\usepackage{listings}

\title{Real Variables: \\
Problem Set III}


\author{
Youngduck Choi \\
Courant Institute of Mathematical Sciences \\
New York University \\
\texttt{yc1104@nyu.edu} \\
}


% The \author macro works with any number of authors. There are two commands
% used to separate the names and addresses of multiple authors: \And and \AND.
%
% Using \And between authors leaves it to \LaTeX{} to determine where to break
% the lines. Using \AND forces a linebreak at that point. So, if \LaTeX{}
% puts 3 of 4 authors names on the first line, and the last on the second
% line, try using \AND instead of \And before the third author name.

\newcommand{\fix}{\marginpar{FIX}}
\newcommand{\new}{\marginpar{NEW}}

\nipsfinalcopy % Uncomment for camera-ready version

\begin{document}


\maketitle

\begin{abstract}
This work contains solutions to the problem set III of Real Variables 2015 at NYU.
\end{abstract}

\section{Solutions}

\bigskip

\begin{question}[1. Royden 3.20]
\end{question}
\begin{solution}
Let $A$ and $B$ be any sets. The LHS of the first equation can be written as 
\eQb
\chi_{A \cap B} &=& 
\begin{cases} 1 &\mbox{if } x \in A \cap B \\ 
0 & \mbox{if } x \notin A \cap B. \end{cases} \\ 
\eQe
By noting that the product of has to be of the form, $1\cdot 1$, to yield $1$,
the RHS of the second equation can be written as 
\eQb
\chi_{A}\chi_{B} &=&
\begin{cases} 1 &\mbox{if } x \in A  \mbox{ and if } x \in B \\ 
0 & \mbox{ otherwise, }\end{cases} \\ 
\eQe
which is precisely the LHS as $x \in A \text{ and } x \in B$ is the definition of 
$x \in A \cap B$. Now, the LHS of the second equation can be written as 
\eQb
\chi_{A \cup B} &=& \begin{cases} 1 &\mbox{if } x \in A \cup B \\ 
0 & \mbox{if } x \notin A \cup B. \end{cases} \\
\eQe
The RHS of the second equation can be written as 
\eQb
\chi_{A} + \chi_{B} - \chi_{A} \cdot \chi_{B} &=& 
\begin{cases} 1 + 1 - 1 \cdot 1 = 1 &\mbox{if } x \in A, x \in B \\ 
1 + 0 - 1 \cdot 0 = 1& \mbox{if } x \in A, x \notin B  \\
0 + 1 - 0 \cdot 1 = 1& \mbox{if } x \notin A, x \in B  \\
0 + 0 - 0 \cdot 0 = 0& \mbox{if } x \notin A, x \notin B, \end{cases} \\
\eQe
which can be simplified to 
\eQb
\chi_{A} + \chi_{B} - \chi_{A} \cdot \chi_{B} &=& 
\begin{cases} 1 &\mbox{if } x \in A \cup B \\ 
0 & \mbox{if } x \notin A \cup B \end{cases}, \\
\eQe
as desired. The LHS of the third equation can be written as
\eQb
\chi_{A^c} &=& \begin{cases} 1 & \mbox{if } x \in A^c \\ 
0 & \mbox{if } x \notin A^c \end{cases} \\,
\eQe
as desired. The RHS of the third equation can be written as
\eQb
1-\chi_{A} &=& \begin{cases} 0 & \mbox{if } x \in A \\ 
1 & \mbox{if } x \notin A \end{cases}, \\
\eQe
which is precisely the LHS, as $x \notin A$ is equivalent to $x \in A^c$ by definition.
Hence, we have shown the three given equalities.
$\qed$
\end{solution}

\bigskip

\begin{question}[2. Royden 3.21]
\end{question}
\begin{solution}
Let $\{f_n \}$ be a sequence of measurable functions with common domain $E$. Consider the

function $\sup \{ f_n \}$, which we will denote as $s$. Let $c \in \mathbb{R}$. 
We wish to show that $\{ x \in E \> | \> s(x) > c\}$ is measurable. By the definition of supremum,
we have that $s(x) > c$ iff there exists $n$ such that $f_n(x) > c$. Hence, it follows that 
\eQb
\{ x \in E \> | \> s(x) > c \} &=& \bigcup_{n=1}^{\infty} 
\{ x \in E \> | \> f_n(x) > c \}. 
\eQe
Since the RHS is a countable collection of measurable sets, the set 
$\{ x \in E \> | \> s(x) > c \}$ is measurable. Since $c$ was arbitrary, $s$ is measurable. 
The $\inf$ case can be shown analogously.

\smallskip

Now, consider the $\limsup \{ f_n \} $ case. 
Observe that $\underset{n \to \infty}{\limsup}f_n = \inf_{n} 
\sup_{m \geq n}f_n$. Consequently, as we have shown that $\sup\{f_n\}$ and $\inf\{f_n\}$ are 
measurable functions, we have that $\limsup \{ f_n \}$ is measurable. The $\liminf$ case can 
be shown analogously. $\qed$
\end{solution}

\bigskip

\begin{question}[3. Royden 3.27]
\end{question}
\begin{solution}
Let $f = 1$ on $[0,\infty)$. Define $f_n = \chi_{[0,n]}$ for all $n$. Then, we have that
$f_n \to f$ pointwise everywhere. Suppose for sake of contradiction that there exists a 
closed set $F$ such that $m([0,\infty) \setminus F) < \epsilon$ and $f_n \to f$ uniformly on $F$. 
$F$ is unbounded, as otherwise $m([0,\infty ) \setminus F) > \infty$, which is a contradiction.
Since $f_n \to f$ uniformly on $F$, there exists $N$ such that $f_n = f$ on $F$. As $F$ is unbounded,
there exists $x \in F\setminus [0,N]$. Since $f_N(x) = 0$ and $f(x) = 1$, this is a contradiction
with $f_n = f$ on $F$. Therefore, we have shown that the conclusion of Egoroff can fail
without the finiteness assumption on the measure of domain. $\qed$
\end{solution}

\bigskip

\begin{question}[4. Royden 4.12]
\end{question}
\begin{solution}
Let $f$ a bounded measurable function on a set of finite measure $E$. Assume $g$ is bounded
and $f = g$ a.e. on $E$. First, as $g$ is a function that equals a measurable function a.e.,
we have that $g$ is measurable. Since both $f$ and $g$
are bounded measurable functions, we have $\int_{E} f$ and $\int_{E} g$ terms well-defined.
Let $E_0 = \{ x \in E \> | \> f(x) \neq g(x) \}$. Note that 
$\mathrm{m}(E_0) = 0$, as $f=g$ a.e. Consequently, $E \setminus E_0$ and $E_0$ are 
disjoint measurable sets.  Then, by additivity over domain and linearity of integration,
we have 
\eQb
\left| \int_{E}f - \int_{E} g \right| &=& 
\left| \int_{E \setminus E_0}f - \int_{E \setminus E_0}g 
 + \int_{E_0}f - \int_{E_0}g \right| \\
&=& \left| \int_{E \setminus E_0} f - g + 
\int_{E_0} f - g \right|.
\eQe
As $f = g$ on $E \setminus E_0$, we have
\eQb
\left| \int_{E}f - \int_{E} g \right| &=&  
\left| \int_{E_0} f - g \right| \\
&\leq& \int_{E_0} \left| f - g \right|.
\eQe
As both $f$ and $g$ are bounded, there exists $M$ such that $|f-g| \leq M$ on $E_0$. Hence, we have
\eQb
\left| \int_{E}f - \int_{E} g \right|  
&\leq& M \cdot \mathrm{m}(E_0) \\
&\leq& 0.
\eQe 
Therefore, we have $\int_{E} f = \int_{E} g$ 
as desired. $\qed$
\end{solution}

\bigskip

\begin{question}[5. Royden 4.23]
\end{question}
\begin{solution}
Let $\{ a_n \}$ be a sequence of non-negative real numbers. Let $f$ be a function on 
$E = [1,\infty )$, defined by setting $f(x) = a_n$ if $ n \leq x < n+1$. 
Then, consider the following sequence of functions of 
nonnegative real numbers $\{ f_n \}$ defined on $E$ such that 
\eQb
f_n = \sum_{k=1}^{n} a_k \chi_{I_k},
\eQe
where $I_k$ denotes the characteristic function of an interval $[k,k+1)$. 
Notice that $\{ f_n \}$ is increasing,
and converges to $f$ pointwise everywhere on $E$. Hence,
by the Monotone Convergence Theorem, we have
\eQb
\int_{E} f = \underset{n \to \infty}{\lim} \int_{E} f_n.
\eQe
As the integral on the RHS is a simple function with $n$ values, we have
\eQb
\int_{E} f &=& \underset{n \to \infty}{\lim} \sum_{k=1}^{n} a_k \mathrm{m}(I_k).
\eQe
By noting that $\mathrm{m}(I_k) = 1$ for all $k$ and subsuming the limit into the summation,
we finally obtain
\eQb
\int_{E} f = \sum_{k=1}^{\infty} a_k,
\eQe
as desired. $\qed$
\end{solution}

\bigskip

\begin{question}[6. Royden 4.28]
\end{question}
\begin{solution}
Let $f$ be integrable over $E$ and $C$ a measurable subset of $E$. We wish to show that 
$\int_{C} f = \int_{E} f\cdot \chi_{C}$. First, observe that $f\cdot \chi_{C}$ is measurable.
Furthermore, we have $|f\cdot \chi_{C}| \leq f $ on $E$. Hence, by the integral comparison test,
we have that $f \cdot \chi_{C}$ is integrable over $E$. It follows that 
\eQb
\int_{E}f \cdot \chi_{C} &=& \int_{E} (f \cdot \chi_{C})^{+} - \int_{E} (f \cdot \chi_{C})^{-}. 
\eQe
By the additivity over domain of integration for nonnegative measurable functions, we have
\eQb
\int_{E}f \cdot \chi_{C} &=& \int_{E\setminus C} (f \cdot \chi_{C})^{+} + \int_{C} 
(f \cdot \chi_{C})^{+} \\ 
&-& \int_{E\setminus C} (f \cdot \chi_{C})^{-} - 
\int_{C} (f \cdot \chi_{C})^{-}. \\ 
\eQe
We can write $(f \cdot \chi_{C})^{+}$ and $(f \cdot \chi_{C})^{-}$ explicitly as follow:
\eQb
(f \cdot \chi_{C})^{+} &=& \begin{cases} \max(f(x),0) &\mbox{if } x \in C \\ 
0 & \mbox{if } x \notin C \end{cases} \\
(f \cdot \chi_{C})^{-} &=& \begin{cases} \max(-f(x),0) &\mbox{if } x \in C \\ 
0 & \mbox{if } x \notin C. \end{cases} \\
\eQe 
Hence, the above integral can be simplified to 
\eQb
\int_{E}f \cdot \chi_{C} &=& \int_{C} (f \cdot \chi_{C})^{+} - \int_{C} 
(f \cdot \chi_{C})^{-}, \\ 
\eQe
which simplifies to 
\eQb
\int_{E}f \cdot \chi_{C} &=& \int_{C} (f \cdot \chi_{C}), 
\eQe
as desired. $\qed$
\end{solution}

\end{document}
