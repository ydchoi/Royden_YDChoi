\documentclass{article} % For LaTeX2e
\usepackage{nips14submit_e,times}
\usepackage{amsmath}
\usepackage{amsthm}
\usepackage{amssymb}
\usepackage{mathtools}
\usepackage{hyperref}
\usepackage{url}
\usepackage{algorithm}
\usepackage[noend]{algpseudocode}
%\documentstyle[nips14submit_09,times,art10]{article} % For LaTeX 2.09

\usepackage{mathrsfs}
\usepackage{graphicx}
\usepackage{caption}
\usepackage{subcaption}

\def\eQb#1\eQe{\begin{eqnarray*}#1\end{eqnarray*}}
\def\eQnb#1\eQne{\begin{eqnarray}#1\end{eqnarray}}
\providecommand{\e}[1]{\ensuremath{\times 10^{#1}}}
\providecommand{\pb}[0]{\pagebreak}


\def\Qb#1\Qe{\begin{question}#1\end{question}}
\def\Sb#1\Se{\begin{solution}#1\end{solution}}

\newenvironment{claim}[1]{\par\noindent\underline{Claim:}\space#1}{}
\newtheoremstyle{quest}{\topsep}{\topsep}{}{}{\bfseries}{}{ }{\thmname{#1}\thmnote{ #3}.}
\theoremstyle{quest}
\newtheorem*{definition}{Definition}
\newtheorem*{theorem}{Theorem}
\newtheorem*{lemma}{Lemma}
\newtheorem*{question}{Question}
\newtheorem*{preposition}{Preposition}
\newtheorem*{exercise}{Exercise}
\newtheorem*{challengeproblem}{Challenge Problem}
\newtheorem*{solution}{Solution}
\newtheorem*{remark}{Remark}
\usepackage{verbatimbox}
\usepackage{listings}

\title{Real Variables: \\
Problem Set IV}


\author{
Youngduck Choi \\
Courant Institute of Mathematical Sciences \\
New York University \\
\texttt{yc1104@nyu.edu} \\
}


% The \author macro works with any number of authors. There are two commands
% used to separate the names and addresses of multiple authors: \And and \AND.
%
% Using \And between authors leaves it to \LaTeX{} to determine where to break
% the lines. Using \AND forces a linebreak at that point. So, if \LaTeX{}
% puts 3 of 4 authors names on the first line, and the last on the second
% line, try using \AND instead of \And before the third author name.

\newcommand{\fix}{\marginpar{FIX}}
\newcommand{\new}{\marginpar{NEW}}

\nipsfinalcopy % Uncomment for camera-ready version

\begin{document}


\maketitle

\begin{abstract}
This work contains solutions to the problem set IV of Real Variables 2015 at NYU.
\end{abstract}

\section{Solutions}

\bigskip

\begin{question}[1. Royden 4.31]
\end{question}
\begin{solution}
\end{solution}

\bigskip

\begin{question}[2. Royden 4.44]
\end{question}
\begin{solution}

\end{solution}

\bigskip

\begin{question}[4. Royden 4.52]
\end{question}
\begin{solution}
\textbf{(a)} Consider the following family of functions:
\eQb
\mathscr{F} = \{ n\chi_{[0,\frac{1}{n}]} 
\}_{n=1}^{\infty}. 
\eQe
Observe that for each $n$, $n\chi_{[0,\frac{1}{n}]}$ is 
integrable and $\int_{0}^{1} |n_\chi{[0,\frac{1}{n}]}| = 1$.
The family $\mathscr{F}$, however, fails to be uniformly integrable.
Fix $\epsilon = \dfrac{1}{2}$. Then, for any $\delta > 0$, by the 
Archimedean property of the reals, there exists $n$, such that
$\dfrac{1}{n} < \delta$. Since the interval $[0,\dfrac{1}{n}]$ 
is measurable, has a measure smaller than $\delta$, and 
$\int_{0}^{\frac{1}{n}} n\chi_{[0,\frac{1}{n}]} = 1 > \dfrac{1}{2}$, 
we have that $\mathscr{F}$ is not uniformly integrable. Hence, by
a counter example, we have shown that under the given assumptions,
the family of functions need not be uniformly integrable. 

\smallskip

\textbf{(b)} We claim that $\mathscr{F}$ with the given assumption
is uniformly integrable. Note that continuity implies integrability.
Fix $\epsilon > 0$. 
Let $f \in \mathscr{F}$. 
Then, for any measurable set $E \subseteq [0,1]$ with
$mE < \delta$ with, by using the $|f| \leq 1$ bound, we obtain 
\eQb
\int_{E} f &\leq& \int_{E} |f| \\
&\leq& \int_{E} 1 \\
&=& mE \\
&\leq& \delta \\
\eQe
By letting $\delta = \epsilon$, we have $\int_{E} f \leq \epsilon$.
Since $\epsilon$ and $f$ were arbitrary, we have shown that
$\mathscr{F}$ is uniformly integrable. 

\smallskip

\textbf{(c)}
Let $\mathscr{F}$ be the family of functions $f$ on $[0,1]$, each
of which is integrable over $[0,1]$ and has $\int_{a}^{b}|f| \leq 
b-a$ for all $[a,b] \subseteq [0,1]$. We claim that
$\mathscr{F}$ is uniformly integrable. 
Fix $\epsilon > 0$ and fix $f \in \mathscr{F}$. 
Let $A \subseteq [0,1]$ be a measurable set such that
$mA < \delta$  By the outer approximation
of measurable set by open sets, there exists an open set $O$ such that
$A \subseteq O$ and $m(O\setminus A) \leq \dfrac{\epsilon}{2}$. Observe
that $O$ can be written as a countable union of disjoint open intervals,
which gives $O = \cup_{i=1}^{\infty} (a_i, b_i)$. 
From the monotonicity and 
excision property of measure, and countable additivity over domain property
of integration, it follows that
\eQb
\int_{A} |f| &\leq& \int_{O} |f| \\
&\leq& \int_{\cup_{i=1}^{\infty}(a_i,b_i)} |f| \\
&=& \sum_{i=1}^{\infty} \int_{(a_i,b_i)} |f| \\
&\leq& \sum_{i=1}^{\infty} \int_{[a_i,b_i])} |f| \\
&\leq& \sum_{i=1}^{\infty} b_i - a_i \\
&=& mO \\
&=& m(O\setminus A) + m(A) \\ 
&\leq& \dfrac{\epsilon}{2} + \delta. 
\eQe
Define $\delta = \dfrac{\epsilon}{2}$ then, we have if 
$A$ is measurable, and $mA <
\delta$, then $\int_{A} |f| < \epsilon$.
Since $\epsilon$ and $f$ were arbitrary, we have that $\mathscr{F}$
is uniformly integrable.  $\qed$

\end{solution}

\end{document}
