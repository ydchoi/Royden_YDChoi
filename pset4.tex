\documentclass{article} % For LaTeX2e
\usepackage{nips14submit_e,times}
\usepackage{amsmath}
\usepackage{amsthm}
\usepackage{amssymb}
\usepackage{mathtools}
\usepackage{hyperref}
\usepackage{url}
\usepackage{algorithm}
\usepackage[noend]{algpseudocode}
%\documentstyle[nips14submit_09,times,art10]{article} % For LaTeX 2.09

\usepackage{mathrsfs}
\usepackage{graphicx}
\usepackage{caption}
\usepackage{subcaption}

\def\eQb#1\eQe{\begin{eqnarray*}#1\end{eqnarray*}}
\def\eQnb#1\eQne{\begin{eqnarray}#1\end{eqnarray}}
\providecommand{\e}[1]{\ensuremath{\times 10^{#1}}}
\providecommand{\pb}[0]{\pagebreak}


\def\Qb#1\Qe{\begin{question}#1\end{question}}
\def\Sb#1\Se{\begin{solution}#1\end{solution}}

\newenvironment{claim}[1]{\par\noindent\underline{Claim:}\space#1}{}
\newtheoremstyle{quest}{\topsep}{\topsep}{}{}{\bfseries}{}{ }{\thmname{#1}\thmnote{ #3}.}
\theoremstyle{quest}
\newtheorem*{definition}{Definition}
\newtheorem*{theorem}{Theorem}
\newtheorem*{lemma}{Lemma}
\newtheorem*{question}{Question}
\newtheorem*{preposition}{Preposition}
\newtheorem*{exercise}{Exercise}
\newtheorem*{challengeproblem}{Challenge Problem}
\newtheorem*{solution}{Solution}
\newtheorem*{remark}{Remark}
\usepackage{verbatimbox}
\usepackage{listings}

\title{Real Variables: \\
Problem Set IV}


\author{
Youngduck Choi \\
Courant Institute of Mathematical Sciences \\
New York University \\
\texttt{yc1104@nyu.edu} \\
}


% The \author macro works with any number of authors. There are two commands
% used to separate the names and addresses of multiple authors: \And and \AND.
%
% Using \And between authors leaves it to \LaTeX{} to determine where to break
% the lines. Using \AND forces a linebreak at that point. So, if \LaTeX{}
% puts 3 of 4 authors names on the first line, and the last on the second
% line, try using \AND instead of \And before the third author name.

\newcommand{\fix}{\marginpar{FIX}}
\newcommand{\new}{\marginpar{NEW}}

\nipsfinalcopy % Uncomment for camera-ready version

\begin{document}


\maketitle

\begin{abstract}
This work contains solutions to the problem set IV of Real Variables 2015 at NYU.
\end{abstract}

\section{Solutions}

\bigskip

\begin{question}[1. Royden 4.31]
\end{question}
\begin{solution}
Let $f$ be a measurable function on $E$, which can be expressed as
$f = g + h$ on $E$, where $g$ is finite and integrable over $E$ and
$h$ is nonnegative. Let $f = g_1 + h_1$ and $f = g_2 + h_2$ satisfying
the given properties of $g$ and $h$ respectively. We wish to show that
\eQb
\int_{E} g_1 + \int_{E} h_1 &=& \int_{E} g_2 + \int_{E} h_2.
\eQe  
Assume that $h_1$ and $h_2$ are integrable. Then, by the linearity of
integration, we have
\eQb
\int_{E} g_1 + \int_{E} h_1 &=& \int_{E} g_1 + h_1 \\
&=& \int_{E} f \\
&=& \int_{E} g_2 + h_2 \\
&=& \int_{E} g_2 + \int_{E} h_2, \\
\eQe
as desired. Now, consider the remaining case of at least one of $h$
not being integrable. Without loss of generality, assume that 
$\int_{E} h_1 = \infty$. Since $g_1 + h_1 = g_2 + h_2$, we have
\eQb
h_2 &=& h_1 + g_1 - g_2 \\
&=& h_1 + (g_1 - g_2)^+ - (g_1 - g_2)^-  \\
&\geq& h_1 - (g_1 - g_2)^- ,
\eQe
with $(g_1 - g_2)^+$ and $(g_1 - g_2)^-$ being properly defined
by the finiteness assumption on the $g$s.
Since $h_2$, $h_1$ and $(g_1 - g_2)^-$ are all non-negative 
measurable functions,
by the monotonicity and linearity of integration of non-negative 
measurable functions,
we have
\eQb
\int_{E} h_2 &\geq& \int_{E} h_1 - (g_1 - g_2 )^- \\ 
&=& \int_{E} h_1 - \int_{E} (g_1 - g_2)^- .
\eQe
From the linearity of general integrable functions, we have that
$g_1 - g_2$ is integrable. Consequently, $(g_1 - g_2)^-$ is integrable 
as well. It follows that
\eQb
\left| \int_{E} (g_1 - g_2)^{-} \right| 
&\leq&  \int_{E}| (g_1 - g_2)^- | \\
&< & \infty .
\eQe
Therefore, we obtain that 
\eQb
\int_{E} h_1 - \int_{E} (g_1 - g_2)^- &=& \infty,
\eQe
which combined with the established inequality of 
$\int_{E} h_2 \geq \int_{E} h_1 - \int_{E} (g_1 - g_2)^-$ yields
\eQb
\int_{E} h_2 = \infty. 
\eQe
Hence, we have
\eQb
\int_{E} g_1 + \int_{E} h_1 &=& \infty \\
&=&  \int_{E} g_2 + \int_{E} h_2,
\eQe
as $g_1$ and $g_2$ are integrable. This completes the proof. $\qed$
\end{solution}

\bigskip

\begin{question}[2. Royden 4.44]
\end{question}
\begin{solution}
Let $f$ be integrable over $\mathbb{R}$ and $\epsilon > 0$. \\
 
\textbf{(i)} 
First, we prove the given property for $f$ nonnegative.
Assume $f \geq 0$. Since $f$ is integrable, thus measuralbe, by
the Simple Approximation Theorem, there exists a sequence of 
increasing simple functions $\{ \phi_n \}$ on $\mathbb{R}$ which
converges pointwise on $\mathbb{R}$ to $f$, such that
\eQb
| \phi_n | &\leq& | f | \text{ on } \mathbb{R},
\eQe 
for all $n$. Now, define a new sequence of simple function by
\eQb
\psi_n &=& max\{ 0, \phi_n \} \cdot \chi_{[-n,n]}.
\eQe
Observe that $\{ \psi_n \}$ is an
increasing sequence of simple functions on $\mathbb{R}$,
which has finite support and is non-negative, that converges to $f$
pointwise. By the Monotone convergnece theorem, there exists $N$ such
that 
\eQb
|\int_{\mathbb{R}} f - \int_{\mathbb{R}} \psi_n | < \epsilon, 
\eQe
for $n \geq N$. By the linearity of integration and the fact that 
$\psi_n \leq f$ for all $n$, we have
\eQb
\epsilon &>& |\int_{\mathbb{R}} f - \int_{\mathbb{R}} \psi_n | \\ 
&=& | \int_{\mathbb{R}} f - \psi_n | \\
&=& \int_{\mathbb{R}} |f - \psi_n|, \\
\eQe
for $n \geq N$. Therefore, we have found a function with the desired 
property, namely $\psi_n$. 
 
\smallskip

Now, we lift the non-negativity constraint. By the definition of
integral, we have
\eQb
\int_{\mathbb{R}} f &=& \int_{\mathbb{R}} f^+ - \int_{\mathbb{R}} f^- . 
\eQe
Since $f$ is integrable, $f^+$ and $f^-$ are integrable and from the
previous result, we have simple functions $\psi^+$ and $\psi^-$
with finite support  
such that
\eQb
\int_{\mathbb{R}} |f^+ - \psi^+ | < \dfrac{\epsilon}{2} \\
\int_{\mathbb{R}} |f^- - \psi^- | < \dfrac{\epsilon}{2}.
\eQe
Observe that $\psi^+ - \psi^-$ is simple and has finite support 
as well. Now, by the triangle inequality and monotonicity of integration,
it follows that
\eQb
\int_{\mathbb{R}} |f - (\psi^+ - \psi^- )|
&=& \int_{\mathbb{R}} |f^+ - f^- - \psi^+ + \psi^- | \\
&\leq& \int_{\mathbb{R}} |f^+ - \psi^+| + |f^- - \psi^- | \\
&=& \int_{\mathbb{R}} |f^+ - \psi^+ | 
+ \int_{\mathbb{R}} |f^- - \psi^-| \\ 
&<& \epsilon.
\eQe
Therefore, $\psi^+ - \psi^-$ is the construction of the function
with the desired property. We have shown that there is a simple function
$\eta$ on $\mathbb{R}$ which has a finite support and $\int_{R} |f - \eta|
< \epsilon$. 

\textbf{(ii)}

\smallskip

\textbf{(iii)}


\end{solution}

\bigskip

\begin{question}[2. Royden 4.47]
\end{question}
\begin{solution}
Let $g$ be integrable over $\mathbb{R}$. \\
\textbf{(i)} Let $k \in \mathbb{R}$.
Assume that $g$ non-negative. Let $E_n = [-n,n]$ and
$E_n - k = [-n-k, n-k]$. By the definition of integration of
non-negative functions, it follows that
\eQb
\int_{E_n} g(x) dx  &=& sup \{ \int_{E_n} h(x) dx \> | \> h
\text{ bounded, measurable, of finite support  and } \\
&& 0 \leq h(x) \leq g(x) \> \text{ for } \> x \in  E_n \} \\
&=& sup \{ \int_{E_n-k} h(x+k) dx \> | \> h
\text{ bounded, measurable, of finite support and } \\
&& 0 \leq h(x+k) \leq g(x+k) \> \text{for } \> x \in E_n-k \} \\
&=& \int_{E_n - k} g(x+k) dx.
\eQe
Since the general integral is defined as the sum of non-negative integrals,
for integrable functions,
the result trivially generalizes to an integrable function. 
From this point on,
we drop the non-negativity assumption on $g$ and assume that $g$ is
integrable. Notice that $\{ E_n \}$ and $\{ E_n - k\}$ form
ascending countable collection of measurable subsets of $\mathbb{R}$, with
$\cup_{n=1}^{\infty} E_n = \cup_{n=1}^{\infty} E_n - k = \mathbb{R}$.
Hence, by the continuity of integration, we obtain
\eQb
\int_{\mathbb{R}} g(x) dx &=& \lim_{n \to \infty} 
\int_{E_n} g(x) dx \\
\int_{\mathbb{R}} g(x+k) dx &=& \lim_{n \to \infty} 
\int_{E_n-k} g(x+k) dx. \\
\eQe 
Since $\int_{E_n} g(x) dx = \int_{E_n-k} g(x+k) dx$ for all $n$,
it follow that
\eQb
\int_{\mathbb{R}} g(x) dx &=&  
\int_{\mathbb{R}} g(x+k) dx, \\ 
\eQe
as desired. $\qed$

\smallskip

\textbf{(ii)}
dd

\end{solution}


\bigskip

\begin{question}[4. Royden 4.52]
\end{question}
\begin{solution}
\textbf{(a)} Consider the following family of functions:
\eQb
\mathscr{F} = \{ n\chi_{[0,\frac{1}{n}]} 
\}_{n=1}^{\infty}. 
\eQe
Observe that for each $n$, $n\chi_{[0,\frac{1}{n}]}$ is 
integrable and $\int_{0}^{1} |n_\chi{[0,\frac{1}{n}]}| = 1$.
The family $\mathscr{F}$, however, fails to be uniformly integrable.
Fix $\epsilon = \dfrac{1}{2}$. Then, for any $\delta > 0$, by the 
Archimedean property of the reals, there exists $n$, such that
$\dfrac{1}{n} < \delta$. Since the interval $[0,\dfrac{1}{n}]$ 
is measurable, has a measure smaller than $\delta$, and 
$\int_{0}^{\frac{1}{n}} n\chi_{[0,\frac{1}{n}]} = 1 > \dfrac{1}{2}$, 
we have that $\mathscr{F}$ is not uniformly integrable. Hence, by
a counter example, we have shown that under the given assumptions,
the family of functions need not be uniformly integrable. 

\smallskip

\textbf{(b)} We claim that $\mathscr{F}$ with the given assumption
is uniformly integrable. Note that continuity implies integrability.
Fix $\epsilon > 0$. 
Let $f \in \mathscr{F}$. 
Then, for any measurable set $E \subseteq [0,1]$ with
$mE < \delta$ with, by using the $|f| \leq 1$ bound, we obtain 
\eQb
\int_{E} f &\leq& \int_{E} |f| \\
&\leq& \int_{E} 1 \\
&=& mE \\
&\leq& \delta \\
\eQe
By letting $\delta = \epsilon$, we have $\int_{E} f \leq \epsilon$.
Since $\epsilon$ and $f$ were arbitrary, we have shown that
$\mathscr{F}$ is uniformly integrable. 

\smallskip

\textbf{(c)}
Let $\mathscr{F}$ be the family of functions $f$ on $[0,1]$, each
of which is integrable over $[0,1]$ and has $\int_{a}^{b}|f| \leq 
b-a$ for all $[a,b] \subseteq [0,1]$. We claim that
$\mathscr{F}$ is uniformly integrable. 
Fix $\epsilon > 0$ and fix $f \in \mathscr{F}$. 
Let $A \subseteq [0,1]$ be a measurable set such that
$mA < \delta$  By the outer approximation
of measurable set by open sets, there exists an open set $O$ such that
$A \subseteq O$ and $m(O\setminus A) \leq \dfrac{\epsilon}{2}$. Observe
that $O$ can be written as a countable union of disjoint open intervals,
which gives $O = \cup_{i=1}^{\infty} (a_i, b_i)$. 
From the monotonicity and 
excision property of measure, and countable additivity over domain property
of integration, it follows that
\eQb
\int_{A} |f| &\leq& \int_{O} |f| \\
&\leq& \int_{\cup_{i=1}^{\infty}(a_i,b_i)} |f| \\
&=& \sum_{i=1}^{\infty} \int_{(a_i,b_i)} |f| \\
&\leq& \sum_{i=1}^{\infty} \int_{[a_i,b_i]} |f| \\
&\leq& \sum_{i=1}^{\infty} b_i - a_i \\
&=& mO \\
&=& m(O\setminus A) + m(A) \\ 
&\leq& \dfrac{\epsilon}{2} + \delta. 
\eQe
Define $\delta = \dfrac{\epsilon}{2}$ then, we have if 
$A$ is measurable, and $mA <
\delta$, then $\int_{A} |f| < \epsilon$.
Since $\epsilon$ and $f$ were arbitrary, we have that $\mathscr{F}$
is uniformly integrable.  $\qed$

\end{solution}

\smallskip

\begin{question}[5. 5-11]
\end{question}
\begin{solution}
Assume that $E$ has finite measure. Let $\{ f_n \}$ be a sequence of
measurable functions on $E$ and $f$ a measurable on $E$ for which
$f$ and each $f_n$ is finite a.e. 

We first show that the if implication holds by proving its
contrapositive. Assume that $f_n$ does not converge to $f$ in measure. 
This implies that there exists $\delta > 0$ and $\eta > 0$ such that
\eQb
m\{x \in E \> | \> |f_n(x) - f(x)| > \eta \} > \delta,
\eQe
infinitely often in the sequence of $f_n$. Choose $\{ f_{n_k} \}$
such that 
\eQb
m\{ x \in E \> | \> |f_{n_k}(x) - f(x) | > \eta \} > \delta ,
\eQe
for all $k$. Hence, there is a non measure-zero set, on which
$|f_{n_k} - f| > \delta$ for all $k$. Hence, 
any subsequence of $\{ f_{n_k} \}$ cannot 
converge pointwise a.e. on $E$. Therefore, there exists a subsequence
of $\{f_n\}$ who does not have a further subsequence that converges
to $f$ pointwise a.e. on $E$, which completes the proof. 

\smallskip

Now, we prove the only if implication. Assume that $f_n \to f$ in measure
on $E$. Hence, we have that for all $\eta > 0$,
\eQb
\underset{n \to \infty}{\lim} m\{ x \in E 
\> | \>  f_n(x) - f(x)| > \eta \}
&\to& 0.
\eQe

\end{solution}

\bigskip

\begin{question}[6. 5-13]
\end{question}
\begin{solution}
dd
\end{solution}





\end{document}
